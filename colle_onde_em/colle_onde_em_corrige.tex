 \documentclass{report}
 
\usepackage[utf8]{inputenc} 
\usepackage[T1]{fontenc}      
\usepackage[top=2.0cm, bottom=3cm, left=3.0cm, right=3.0cm]{geometry}
\usepackage{graphicx}
\usepackage{wrapfig}
\usepackage{amsmath,esint }
\usepackage{amssymb} % Pour le sigle euro
\usepackage{bbold} %Pour la matrice identité
\graphicspath{{figures/}{../figures}}

\newcommand*\dif{\mathop{}\!\mathrm{d}}
\newcommand*\diver{\mathop{}\!\mathrm{div}}
\newcommand*\grad{\mathop{}\!\mathrm{grad}}
\newcommand*{\vect}[2]{
	\ensuremath{
    \left\lvert 
      \begin{matrix} 
        #1\\ 
        #2 
      \end{matrix}  
    \right.
  }}

\begin{document}

\section*{Réflexion d'une onde électromagnétique sur des plans métalliques en incidence oblique}

On considère une onde plane progressive se propageant dans le vide, selon le vecteur d'onde $\vec{k}=k\cos\theta\vec{u_x}+k\sin\theta\vec{u_y}$ et à la pulsation $\omega$. Elle arrive sur un plan métallique infiniment conducteur situé sur le demi-espace $x>0$. On notera $\vec{E_i}$ et $\vec{B_i}$ respectivement le champ électrique et le champ magnétique incidents. Le champ électrique est polarisé rectilignement selon $Oz$ et son amplitude est $E_0$.

\begin{itemize}
		
	\item[$\heartsuit$]	Retrouver l'équation de propagation des champs électrique et magnétique. Quelle est la relation de dispersion associée ? 
	
	\item[$\heartsuit$] Expliciter les expressions des champs $\vec{E_i}$ et $\vec{B_i}$.
	
\end{itemize}

En arrivant sur l'interface, les relations de passage du champ électromagnétique imposent l'apparition d'une onde réfléchie, dont on notera $\vec{E_r}$ et $\vec{B_r}$ les champ électrique et magnétique. On supposera que $\vec{E_r}$ s'écrit sous la forme : 
\begin{align*}
	\vec{E_r}=\vec{E_0'}\exp(i\vec{k_r}\cdot\vec{r}-\omega t)
\end{align*}

\begin{itemize}
	
	\item[$\heartsuit$] Que valent les champs $\vec{E}$ et $\vec{B}$ à l'intérieur de la plaque ? Justifier.
	
	\item[$\heartsuit$] En utilisant les relations de passage, écrire $\vec{E_r}$ en fonction de $E_0$, $k$, $\omega$ et $\theta$. En déduire l'expression du champ magnétique réfléchi, $\vec{B_r}$.
	
	\item[$\heartsuit$] Quelle est alors l'expression du champ électrique $\vec{E}$ résultant pour $x<0$ ? De quel type d'onde s'agit-il ?
	
	\item[$\heartsuit$] 	On place une seconde plaque métallique en $x=-L$. Montrer que la présence de la seconde plaque impose une discrétisation du spectre, c'est-à-dire que seules des fréquences $\omega$ discrètes peuvent se propager pour un angle $\theta$ donné. Tracer les valeurs prises par $\omega$ en fonction de $\theta$. 
	\item[$\heartsuit$] Quelle est la valeur minimale que peut prendre $\omega$ ? Justifier. 
	
	\item[$\heartsuit$] Démontrer que $k_y=\vec{k}\cdot\vec{u_y}$ vérifie l'équation dite de dispersion des modes d'une onde transverse électrique : 
	\begin{equation}
		k_y^2=\frac{\omega^2}{c^2}-\left(\frac{n\pi}{a} \right)^2 
	\end{equation}
	
	\item[$\heartsuit$] Quel est le courant surfacique à la surface de la plaque ?
	
	\item[$\heartsuit$] Calculer l'expression du champ magnétique résultant $\vec{B}$ entre les deux plaques et en déduire l'expression du vecteur de Poyting. Commenter. 
	
\end{itemize}

\newpage

\section*{Propagation d'une onde radio dans un plasma en présence d'un champ magnétique longitudinal}

\begin{itemize}
	
	\item[$\spadesuit$] On écrit le PFD pour un électron du plasma :
	\begin{align*}
		m\vec{a}=-e\vec{v}\wedge\vec{B_{ext}}-e\vec{E}
	\end{align*}
	
	\textit{NB :} dans la force de Lorentz, le champ magnétique de l'OPPM est négligeable par rapport au champ électrique et au champ magnétique extérieur. Pour une puissance solaire de 1kW/m$^2$, on a un champ électrique de $5\cdot10^2$V/m, et donc un champ magnétique associé de 1$\mu$T environ. En comparaison, le champ magnétostatique terrestre est de 50$\mu$T.
	
	On trouve alors comme équation :
	\begin{align*}
		mj\omega\vect{v_x}	{v_y}=-e\vect{E_x}{E_y}-e\vect{v_yB_e}{-v_xB_e}	
	\end{align*}
	On a de plus : $\vec{j}=-en_0\vec{v}$. Pour trouver la relation, il faut inverser la matrice formée par $(v_x,v_y)$ pour exprimer les coordonnées de la vitesse en fonction de $(E_x,E_y)$.
	On trouve : 
	\begin{align*}
	\left\lbrace
	\begin{array}{ccc}
	j_x=\frac{\varepsilon_0\omega_p^2}{\omega^2_c-\omega^2}(j\omega E_x-\omega_cE_y)\\
	\\
	j_y=\frac{\varepsilon_0\omega_p^2}{\omega^2_c-\omega^2}(\omega_cE_x+j\omega E_y)\\
	\end{array}\right.
	\end{align*}	
	
	Cela correspond à une matrice de conductivité :
	\begin{align*}
		\left[\gamma \right] = \frac{\varepsilon_0\omega_p^2}{\omega^2_c-\omega^2}\begin{pmatrix}
			j\omega & -\omega_c \\ 
			\omega_c & j\omega
		\end{pmatrix}
	\end{align*}

	\item[$\spadesuit$] On appelle cette onde polarisation circulaire car si l'on regarde l'orientation du vecteur électrique au cours du temps, elle décrit un cercle.
	
	\item[$\spadesuit$] Les équations de Maxwell permettent d'obtenir rapidement la relation générale suivante :
	\begin{align*}
		\Delta \vec{E} = \mu_0\frac{\partial \vec{j}}{\partial t} +\frac{1}{c^2}\frac{\partial^2 \vec{E}}{\partial t^2}
	\end{align*}
	En introduisant les expressions classique d'OPPM dedans, on trouve :
	\begin{align*}
		\left( \left( k^2-\frac{\omega^2}{c^2}\right)\cdot\mathbb{1} +j\mu_0\omega\left[ \gamma\right] \right) \vec{E}=\vec{0}
	\end{align*}

Si l'on veut obtenir des solutions non-triviales, cad valable pour $\vec{E}\neq\vec{0}$, il faut que :
\begin{align*}
\mathrm{det}\left( \left( k^2-\frac{\omega^2}{c^2}\right)\cdot\mathbb{1} +j\mu_0\omega\left[ \gamma\right] \right)=0
\end{align*}
 cad :
 \begin{align*}
	 k^2-\frac{\omega^2}{c^2}+\frac{\omega_p^2}{c^2}\frac{\omega^2}{\omega^2-\omega_c^2}=\pm\frac{\omega_p^2}{c^2}\frac{\omega\omega_c}{\omega^2-\omega_c^2}
 \end{align*}
 Les deux possibilités correspondent à :
 \begin{align*}
 k^2=\frac{\omega^2}{c^2}\left(1-\frac{\omega_p^2}{\omega(\omega-\omega_c)} \right)\quad,\quad E_y=-jE_x
 \end{align*}
 Cad une onde circulaire gauche et :
  \begin{align*}
 k^2=\frac{\omega^2}{c^2}\left(1-\frac{\omega_p^2}{\omega(\omega+\omega_c)} \right)\quad,\quad E_y=+jE_x
 \end{align*}
 cad une onde polarisée circulaire droite. 

On appelle permittivité relative d'un milieu la quantité complexe $\varepsilon_r$ que l'on peut définir ici à travers la relation $k^2=\varepsilon_r\varepsilon_0\mu_0\omega^2=\varepsilon_r\omega^2/c^2$. 

	\item[$\spadesuit$] On trouve $\varepsilon_{rg} = 1-\frac{\omega_p^2}{\omega(\omega-\omega_c)}$ et $\varepsilon_{rd}=1-\frac{\omega_p^2}{\omega(\omega+\omega_c)}$. Lorsque $\varepsilon_r$ est négatif, $k$ est imaginaire pur et l'onde est évanescente, elle ne se propage pas. 
	
	Les graphes montrent que la propagation de l'onde circulaire droite est possible pour $\omega>\omega_1$, où $\omega_1$ est la pulsation pour laquelle $\varepsilon_{rd}$ devient positif :
	\begin{align*}
		\omega_1=\frac{-\omega_c+\sqrt{\omega_c^2+4\omega_p^2}}{2}
	\end{align*}
Celle de l'onde circulaire gauche est possible pour $\omega<\omega_c$ ou $\omega>\omega_2$, avec :
	\begin{align*}
		\omega_2=\frac{\omega_c+\sqrt{\omega_c^2+4\omega_p^2}}{2}
	\end{align*}
	
	\item[$\spadesuit$] Une onde rectiligne peut être considérée comme une superposition de deux ondes circulaires de même amplitude, tournant dans le même sens. Par exemple, pour une onde dirigée suivant $\vec{e_x}$ :
	\begin{align*}
		\vec{E}(z,t) = E_0\cos(\omega t-kz)\vect{1}{0}=\Re\left(\frac{E_0}{2}\vect{1}{-i}\exp (j\omega t-kz) + \frac{E_0}{2}\vect{1}{i}\exp (j\omega t-kz) \right) 
	\end{align*}
	
	\item[$\spadesuit$] Si on considère que l'onde rentre dans le plasma en $z=0$, on a alors à la sortie :
	\begin{align*}
		\vec{E}(L,t)=\Re\left(\frac{E_0}{2}\vect{1}{-i}\exp (j\omega t-k_gL) + \frac{E_0}{2}\vect{1}{i}\exp (j\omega t-k_dL) \right)
	\end{align*}
	où $k_g$ et $k_d$ sont les vecteurs d'ondes associées aux ondes polarisées circulaire droite et gauche. On trouve que l'onde de sortie est bien rectiligne :
		\begin{align*}
		\vec{E}(L,t)=E_0\vect{\cos\left( \frac{k_g-k_d}{2}L\right) }{\sin\left( \frac{k_g-k_d}{2}L\right) }\times\cos\left( \omega t - \frac{k_g+k_d}{2}L\right) 
	\end{align*}
	La polarisation est donc bien rectiligne avec un angle de décalage $\Psi=\frac{k_g-k_d}{2}L$
	Pour $\lambda_0=30$cm, $\omega=2\pi10^9$rad/s, ce qui est largement supérieur à $\omega_p=5,6\cdot10^7$rad/s. On peut écrire :
	\begin{align*}
		k_g-k_d=\frac{\omega}{c}\left(-\frac{\omega_p^2}{2\omega(\omega-\omega_c)}+ \frac{\omega_p^2}{2\omega(\omega+\omega_c}\right) \simeq \frac{\omega_p^2\omega_c}{\omega^2}\simeq-1,16\cdot10^{-3}\mathrm{rad}
	\end{align*}
	C'est une valeur négligeable. 
	
	
	
\end{itemize}

\newpage

\section*{Ondes électromagnétiques dans un métal conducteur}

On s'intéresse à la propagation des ondes électromagnétiques dans un conducteur métallique, en fonction de leur fréquence et des caractéristiques du métal. Plus particulièrement, on souhaite savoir pourquoi un métal peut être transparent à très basse fréquence, réfléchissant sur sur une certaine bande de fréquences, puis de nouveau transparent à très haute fréquence.

On considère donc que le demi-espace $z>0$ est rempli d'un métal, sur lequel arrive une onde plane progressive monochromatique à la fréquence $\omega$ et polarisée suivant $\vec{e_x}$. 

\begin{itemize}

	\item[$\diamondsuit$] Retrouver l'équation de propagation des ondes électromagnétiques dans le vide. A l'aide des données de l'énoncé, donner l'expression du champ $\vec{E}$ et du champ $\vec{B}$.

\end{itemize}

Pour décrire le métal, on adopte un modèle d'électrons libres, de masse $m$, de charge $-e$ et de densité particulaire $N_0$, soumis au champ électromagnétique, et subissant des collisions en moyenne au bout d'un temps $\tau=1/\omega_c$. On modélise alors le comportement des électrons par l'équation de mouvement (aussi appelé modèle de Drude) :

\begin{equation}
	m\vec{a}=-e\vec{E}-m\frac{\vec{v}}{\tau}
	\label{eq:drude}
\end{equation}

Pour un métal très conducteur, on a $N_0\simeq10^{29}$m$^{-3}$ et $\tau\simeq10^{14}$.

\begin{itemize}

	\item[$\diamondsuit$] En utilisant l'équation \ref{eq:drude}, définir une conductivité $\gamma$ complexe qui dépend de la pulsation $\omega$. Commenter. 
	
	\item[$\diamondsuit$] En utilisant les équations de Maxwell, trouver une équation vérifiée par le champ $\vec{B}$. En déduire une relation de dispersion des OPPM dans le métal. On fera apparaître la pulsation plasma $\omega_p=\sqrt{N_0e^2/m\varepsilon_0}$.
	
	\item[$\diamondsuit$] Comparer $\omega_c=1/\tau$ et $\omega_p$. Justifier de l'existence de 3 régimes de propagation dans le métal que nous allons étudier par la suite.

\end{itemize}

\textbf{On se place dans le cas où $\omega\ll\omega_c$.}

\begin{itemize}

	\item[$\diamondsuit$] Que devient la relation de dispersion dans ce cas-là ? Trouver les solutions possibles pour $k$.

	\item[$\diamondsuit$] Écrire l'expression du champ $\vec{E}$ dans le métal, puis celle du champ $\vec{B}$. Comment appelle t-on ce régime et le phénomène associé ? 

\end{itemize}

\textbf{On se place dans le cas où $\omega\gg\omega_c$.}

\begin{itemize}

	\item[$\diamondsuit$] Que devient la relation de dispersion dans ce cas-là ? Trouver les solutions possibles pour $k$.
	
	\item[$\diamondsuit$] Écrire l'expression du champ $\vec{E}$ dans le métal, puis celle du champ $\vec{B}$. On distinguera les cas $\omega<\omega_p$ et $\omega>\omega_p$. Décrire alors le comportement de l'onde dans ces 2 situations.

\end{itemize}

\textbf{Bilan}

\begin{itemize}

	\item[$\diamondsuit$] Finalement, résumer les 3 situations rencontrées et justifier les observations décrites dans l'énoncé.

	\item[$\diamondsuit$] Quelle est la différence fondamentale entre un plasma vu en cours et un métal comme décrit ici ?

\end{itemize}

\end{document}