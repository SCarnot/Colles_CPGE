 \documentclass{report}
 
\usepackage[utf8]{inputenc} 
\usepackage[T1]{fontenc}      
\usepackage[top=2.0cm, bottom=3cm, left=3.0cm, right=3.0cm]{geometry}
\usepackage{graphicx}
\usepackage{wrapfig}
\usepackage{amsmath,esint }
\usepackage{amssymb} % Pour le sigle euro
\usepackage{bbold} %Pour la matrice identité
\graphicspath{{figures/}{../figures}}

\newcommand*\dif{\mathop{}\!\mathrm{d}}
\newcommand*\diver{\mathop{}\!\mathrm{div}}
\newcommand*\grad{\mathop{}\!\mathrm{grad}}
\newcommand*{\vect}[2]{
	\ensuremath{
    \left\lvert 
      \begin{matrix} 
        #1\\ 
        #2 
      \end{matrix}  
    \right.
  }}

\begin{document}

\section*{Réflexion d'une onde électromagnétique sur des plans métalliques en incidence oblique}

On considère une onde plane progressive se propageant dans le vide, selon le vecteur d'onde $\vec{k}=k\cos\theta\vec{u_x}+k\sin\theta\vec{u_y}$ et à la pulsation $\omega$. Elle arrive sur un plan métallique infiniment conducteur situé sur le demi-espace $x>0$. On notera $\vec{E_i}$ et $\vec{B_i}$ respectivement le champ électrique et le champ magnétique incidents. Le champ électrique est polarisé rectilignement selon $Oz$ et son amplitude est $E_0$.

\begin{itemize}
		
	\item[$\heartsuit$]	Retrouver l'équation de propagation des champs électrique et magnétique. Quelle est la relation de dispersion associée ? 
	
	\item[$\heartsuit$] Expliciter les expressions des champs $\vec{E_i}$ et $\vec{B_i}$.
	
\end{itemize}

En arrivant sur l'interface, les relations de passage du champ électromagnétique imposent l'apparition d'une onde réfléchie, dont on notera $\vec{E_r}$ et $\vec{B_r}$ les champ électrique et magnétique. On supposera que $\vec{E_r}$ s'écrit sous la forme : 
\begin{align*}
	\vec{E_r}=\vec{E_0'}\exp(i\vec{k_r}\cdot\vec{r}-\omega t)
\end{align*}

\begin{itemize}
	
	\item[$\heartsuit$] Que valent les champs $\vec{E}$ et $\vec{B}$ à l'intérieur de la plaque ? Justifier.
	
	\item[$\heartsuit$] En utilisant les relations de passage, écrire $\vec{E_r}$ en fonction de $E_0$, $k$, $\omega$ et $\theta$. En déduire l'expression du champ magnétique réfléchi, $\vec{B_r}$.
	
	\item[$\heartsuit$] Quelle est alors l'expression du champ électrique $\vec{E}$ résultant pour $x<0$ ? De quel type d'onde s'agit-il ?
	
	\item[$\heartsuit$] 	On place une seconde plaque métallique en $x=-L$. Montrer que la présence de la seconde plaque impose une discrétisation du spectre, c'est-à-dire que seules des fréquences $\omega$ discrètes peuvent se propager pour un angle $\theta$ donné. Tracer les valeurs prises par $\omega$ en fonction de $\theta$. 
	\item[$\heartsuit$] Quelle est la valeur minimale que peut prendre $\omega$ ? Justifier. 
	
	\item[$\heartsuit$] Démontrer que $k_y=\vec{k}\cdot\vec{u_y}$ vérifie l'équation dite de dispersion des modes d'une onde transverse électrique : 
	\begin{equation}
		k_y^2=\frac{\omega^2}{c^2}-\left(\frac{n\pi}{a} \right)^2 
	\end{equation}
	
	\item[$\heartsuit$] Quel est le courant surfacique à la surface de la plaque ?
	
	\item[$\heartsuit$] Calculer l'expression du champ magnétique résultant $\vec{B}$ entre les deux plaques et en déduire l'expression du vecteur de Poyting. Commenter. 
	
\end{itemize}

\newpage

\section*{Propagation d'une onde radio dans un plasma en présence d'un champ magnétique longitudinal}

\begin{itemize}
	
	\item[$\spadesuit$] On écrit le PFD pour un électron du plasma :
	\begin{align*}
		m\vec{a}=-e\vec{v}\wedge\vec{B_{ext}}-e\vec{E}
	\end{align*}
	
	\textit{NB :} dans la force de Lorentz, le champ magnétique de l'OPPM est négligeable par rapport au champ électrique et au champ magnétique extérieur. Pour une puissance solaire de 1kW/m$^2$, on a un champ électrique de $5\cdot10^2$V/m, et donc un champ magnétique associé de 1$\mu$T environ. En comparaison, le champ magnétostatique terrestre est de 50$\mu$T.
	
	On trouve alors comme équation :
	\begin{align*}
		mj\omega\vect{v_x}	{v_y}=-e\vect{E_x}{E_y}-e\vect{v_yB_e}{-v_xB_e}	
	\end{align*}
	On a de plus : $\vec{j}=-en_0\vec{v}$. Pour trouver la relation, il faut inverser la matrice formée par $(v_x,v_y)$ pour exprimer les coordonnées de la vitesse en fonction de $(E_x,E_y)$.
	On trouve : 
	\begin{align*}
	\left\lbrace
	\begin{array}{ccc}
	j_x=\frac{\varepsilon_0\omega_p^2}{\omega^2_c-\omega^2}(j\omega E_x-\omega_cE_y)\\
	\\
	j_y=\frac{\varepsilon_0\omega_p^2}{\omega^2_c-\omega^2}(\omega_cE_x+j\omega E_y)\\
	\end{array}\right.
	\end{align*}	
	
	Cela correspond à une matrice de conductivité :
	\begin{align*}
		\left[\gamma \right] = \frac{\varepsilon_0\omega_p^2}{\omega^2_c-\omega^2}\begin{pmatrix}
			j\omega & -\omega_c \\ 
			\omega_c & j\omega
		\end{pmatrix}
	\end{align*}

	\item[$\spadesuit$] On appelle cette onde polarisation circulaire car si l'on regarde l'orientation du vecteur électrique au cours du temps, elle décrit un cercle.
	
	\item[$\spadesuit$] Les équations de Maxwell permettent d'obtenir rapidement la relation générale suivante :
	\begin{align*}
		\Delta \vec{E} = \mu_0\frac{\partial \vec{j}}{\partial t} +\frac{1}{c^2}\frac{\partial^2 \vec{E}}{\partial t^2}
	\end{align*}
	En introduisant les expressions classique d'OPPM dedans, on trouve :
	\begin{align*}
		\left( \left( k^2-\frac{\omega^2}{c^2}\right)\cdot\mathbb{1} +j\mu_0\omega\left[ \gamma\right] \right) \vec{E}=\vec{0}
	\end{align*}

Si l'on veut obtenir des solutions non-triviales, cad valable pour $\vec{E}\neq\vec{0}$, il faut que :
\begin{align*}
\mathrm{det}\left( \left( k^2-\frac{\omega^2}{c^2}\right)\cdot\mathbb{1} +j\mu_0\omega\left[ \gamma\right] \right)=0
\end{align*}
 cad :
 \begin{align*}
	 k^2-\frac{\omega^2}{c^2}+\frac{\omega_p^2}{c^2}\frac{\omega^2}{\omega^2-\omega_c^2}=\pm\frac{\omega_p^2}{c^2}\frac{\omega\omega_c}{\omega^2-\omega_c^2}
 \end{align*}
 Les deux possibilités correspondent à :
 \begin{align*}
 k^2=\frac{\omega^2}{c^2}\left(1-\frac{\omega_p^2}{\omega(\omega-\omega_c)} \right)\quad,\quad E_y=-jE_x
 \end{align*}
 Cad une onde circulaire gauche et :
  \begin{align*}
 k^2=\frac{\omega^2}{c^2}\left(1-\frac{\omega_p^2}{\omega(\omega+\omega_c)} \right)\quad,\quad E_y=+jE_x
 \end{align*}
 cad une onde polarisée circulaire droite. 

On appelle permittivité relative d'un milieu la quantité complexe $\varepsilon_r$ que l'on peut définir ici à travers la relation $k^2=\varepsilon_r\varepsilon_0\mu_0\omega^2=\varepsilon_r\omega^2/c^2$. 

	\item[$\spadesuit$] On trouve $\varepsilon_{rg} = 1-\frac{\omega_p^2}{\omega(\omega-\omega_c)}$ et $\varepsilon_{rd}=1-\frac{\omega_p^2}{\omega(\omega+\omega_c)}$. Lorsque $\varepsilon_r$ est négatif, $k$ est imaginaire pur et l'onde est évanescente, elle ne se propage pas. 
	
	Les graphes montrent que la propagation de l'onde circulaire droite est possible pour $\omega>\omega_1$, où $\omega_1$ est la pulsation pour laquelle $\varepsilon_{rd}$ devient positif :
	\begin{align*}
		\omega_1=\frac{-\omega_c+\sqrt{\omega_c^2+4\omega_p^2}}{2}
	\end{align*}
Celle de l'onde circulaire gauche est possible pour $\omega<\omega_c$ ou $\omega>\omega_2$, avec :
	\begin{align*}
		\omega_2=\frac{\omega_c+\sqrt{\omega_c^2+4\omega_p^2}}{2}
	\end{align*}
	
	\item[$\spadesuit$] Une onde rectiligne peut être considérée comme une superposition de deux ondes circulaires de même amplitude, tournant dans le même sens. Par exemple, pour une onde dirigée suivant $\vec{e_x}$ :
	\begin{align*}
		\vec{E}(z,t) = E_0\cos(\omega t-kz)\vect{1}{0}=\Re\left(\frac{E_0}{2}\vect{1}{-i}\exp (j\omega t-kz) + \frac{E_0}{2}\vect{1}{i}\exp (j\omega t-kz) \right) 
	\end{align*}
	
	\item[$\spadesuit$] Si on considère que l'onde rentre dans le plasma en $z=0$, on a alors à la sortie :
	\begin{align*}
		\vec{E}(L,t)=\Re\left(\frac{E_0}{2}\vect{1}{-i}\exp (j\omega t-k_gL) + \frac{E_0}{2}\vect{1}{i}\exp (j\omega t-k_dL) \right)
	\end{align*}
	où $k_g$ et $k_d$ sont les vecteurs d'ondes associées aux ondes polarisées circulaire droite et gauche. On trouve que l'onde de sortie est bien rectiligne :
		\begin{align*}
		\vec{E}(L,t)=E_0\vect{\cos\left( \frac{k_g-k_d}{2}L\right) }{\sin\left( \frac{k_g-k_d}{2}L\right) }\times\cos\left( \omega t - \frac{k_g+k_d}{2}L\right) 
	\end{align*}
	La polarisation est donc bien rectiligne avec un angle de décalage $\Psi=\frac{k_g-k_d}{2}L$
	Pour $\lambda_0=30$cm, $\omega=2\pi10^9$rad/s, ce qui est largement supérieur à $\omega_p=5,6\cdot10^7$rad/s. On peut écrire :
	\begin{align*}
		k_g-k_d=\frac{\omega}{c}\left(-\frac{\omega_p^2}{2\omega(\omega-\omega_c)}+ \frac{\omega_p^2}{2\omega(\omega+\omega_c}\right) \simeq \frac{\omega_p^2\omega_c}{\omega^2}\simeq-1,16\cdot10^{-3}\mathrm{rad}
	\end{align*}
	C'est une valeur négligeable. 
	
	
	
\end{itemize}

\newpage

\section*{Ondes électromagnétiques dans un métal conducteur}

\begin{itemize}

	\item[$\diamondsuit$] Equation de propagation d'Alembert, archi-classique. Avec les données de l'énoncé, on a dans le vide :
		\begin{align*}
		\begin{cases}
        \vec{E}&=E_0\exp[i(kz-\omega t)]\vec{e}_x \\
		\vec{B}&=\frac{E_0}{c}\exp[i(kz-\omega t)]\vec{e}_y
		\end{cases}  
		\end{align*}

	\item[$\diamondsuit$] En régime sinusoïdal, l'équation du mouvement des électrons devient :
	\begin{align*}
		mi\omega\vec{v}=-e\vec{E}-\frac{m}{\tau}\vec{v}
	\end{align*}
	Sachant que $\vec{j}=-eN_0$, on a donc :
	\begin{align*}
		\vec{j}=\gamma\vec{E}=\frac{\varepsilon_0\omega_p^2}{i\omega+\frac{1}{\tau}}\vec{E}
	\end{align*}
	Ou encore :
	\begin{align*}
		\gamma=\frac{\gamma_0}{1+i\omega\tau}
	\end{align*}
	avec $\gamma_0=\frac{N_0e^2\tau}{m}$ est la conductivité statique du métal.
	
	\item[$\diamondsuit$] Les équations de Maxwell donnent :
	\begin{align*}
		\Delta \vec{E} = \mu_0\frac{\partial \vec{j}}{\partial t}+\varepsilon_0\mu_0\frac{\partial^2\vec{E}}{\partial t^2}
	\end{align*}
	
	On a alors :
	\begin{align*}
			k^2 &= \frac{\omega^2}{c^2}\left(1-i\frac{\gamma}{\varepsilon_0\omega} \right) \\
		&= \frac{\omega^2}{c^2}\left(1-i\frac{i \omega_p^2}{\omega\left( i\omega+\frac{1}{\tau}\right) } \right)
	\end{align*}
	
	\item[$\diamondsuit$] Comparer $\omega_c=1/\tau\simeq10^{14}$ et $\omega_p\simeq10^{16}$. On a donc $\omega_c\ll\omega_p$. Il existe bien trois régimes : 
	\begin{itemize}
		\item[1 - ] $\omega\ll \omega_c \ll\omega_p$. Régime basse fréquence.
		\item[2 - ] $\omega_c\ll \omega < \omega_p$. Régime "moyenne" fréquence.
		\item[3 - ] $\omega_c \ll \omega_p < \omega$. Régime haute fréquence.
	\end{itemize}

\end{itemize}

\textbf{On se place dans le cas où $\omega\ll\omega_c$.}

\begin{itemize}

	\item[$\diamondsuit$] On trouve alors que :
	\begin{align*}
		k^2=-\frac{i\omega_p^2\omega}{\omega_cc^2}=-i\gamma_0\mu_0\omega
	\end{align*}

	\item[$\diamondsuit$] Dans ce cas-là :
	\begin{align*}
		k = \frac{1-j}{\delta}
	\end{align*}
	avec $\delta=\sqrt{2/\mu_0\gamma\omega}$.
	Et :
	\begin{align}
     	\vec{E}=E_{0}\exp[-z/\delta]\exp[i(\omega t-z/\delta)]\vec{e}_x
	\end{align}
\end{itemize}
On retrouve bien le cas de l'effet de peau à basse fréquence. L'onde se propage encore dans le métal mais son amplitude décroit exponentiellement.

\textbf{On se place dans le cas où $\omega\gg\omega_c$.}

\begin{itemize}

	\item[$\diamondsuit$] Dans ce cas-là, c'est le terme dissipatif en $m\vec{v}/\tau$ qui devient négligeable, et on se retrouve dans le cas du plasma non dissipatif :
	\begin{align*}
		k^2=\frac{\omega^2}{c^2}\left( 1-\frac{\omega_p^2}{\omega^2}\right) 
	\end{align*}
	
	\item[$\diamondsuit$] Dans ce cas plasma, la conductivité devient imaginaire pure, donc il n'y a plus de dissipation (le terme $\vec{j}\cdot\vec{E}$ est en $\cos(\omega t)\sin(\omega t)$ et sa valeur moyenne s'annule).
	
	Si $\omega<\omega_p$, $k$ est imaginaire pur : $k=i\delta$ et alors :
	\begin{align}
     	\vec{E}=E_{0}\exp[-z/\delta]\exp[i\omega t]\vec{e}_x
	\end{align}
	avec $\delta = \frac{c^2}{\omega_p^2-\omega}$. L'onde est stationnaire et evanescente : elle ne se propage plus du tout, contrairement au cas de l'effet de peau. 
	
	Si $\omega>\omega_p$, $k$ est réel : 
	\begin{align}
     	\vec{E}=E_{0}\exp[-z/\delta]\exp[i(\omega t-kz)]\vec{e}_x
	\end{align}
L'onde se propage a une vitesse différente que celle-dans le vide mais le métal est transparent pour l'onde.

\end{itemize}

\textbf{Bilan}

\begin{itemize}

	\item[$\diamondsuit$] Finalement, résumer les 3 situations rencontrées et justifier les observations décrites dans l'énoncé.


\end{itemize}

\section*{Guide d'onde métallique}

\begin{itemize}

	\item[$\bigstar$] Ce n'est pas une OPPM : l'amplitude n'est pas uniforme dans le plan orthogonal à la direction de propagation. On ne peut donc pas utiliser la relation de structure ; la relation de dispersion du vide $\omega=kc$ est \textit{a priori} non valide.
	
	\item[$\bigstar$] On utilise $div(\vec{E}=0$, et on voit rapidement que :
	\begin{align*}
		\frac{\partial f}{\partial x} =0 
	\end{align*}
	Avec les autres équations de Maxwell, on trouve très facilement l'équation d'Alembert sur $\vec{E}$ dans le guide, qui est assimilé au vide. La présence des parois modifient les conditions aux limites. En injectant l'expression de $\vec{E}$ dans l'équation de propagation, on trouve :
	\begin{align*}
		f''(y)+\left(\frac{\omega^2}{c^2} - k^2 \right)\cdot f(y) =0
	\end{align*}
	
	Si $\frac{\omega^2}{c^2} - k^2<0$, alors on a des solutions exponentielles :
	\begin{align*}
		f(y)=A\exp(Ky)+B\exp(-Ky)
	\end{align*}
	avec $K=-\frac{\omega^2}{c^2} + k^2$
	
	Si $\frac{\omega^2}{c^2} - k^2>0$, alors on a des solutions sinusoïdales :
	\begin{align*}
		f(y)=A\cos(Ky)+B\sin(-Ky)
	\end{align*}
	avec $K=\frac{\omega^2}{c^2} - k^2$
	
	\item[$\bigstar$] Les conditions aux limites sont :
	\begin{align*}
		E_x(y=0)=E(y=b)=0
	\end{align*}
	Elles proviennent des relations de passage à l'interface avec la paroi métallique, où la composante tangentielle du champ électrique est continue. Le champ électrique étant nul dans le métal infiniment conducteur, la composante tangentielle de celui-ci dans le vide doit être nulle.
	Dans le cas des solutions exponentielles, on aurait :
	\begin{align}
    	\begin{cases}
        A+B  =0\\
     	\exp(Kb)+\exp(-Kb) = 0
    \end{cases}  
	\end{align}
	La solution serait alors $A=B=0$. A l'inverse, des solutions sinusoïdales non nulles sont possibles avec ces conditions aux limites, où l'on a $B=0$ et $A=E_0$, l'amplitude du champ éléctrique :
	\begin{align*}
		\vec{E}=E_0\sin(Kx)\exp[i(kz-\omega t)]\vec{e}_x
	\end{align*}
	
	\item[$\bigstar$]	Les solution ssinusoïdales trouvées correspondent au cas $K^2=\omega^2/c^2-k^2>0$, cad $\lambda>2\pi c/\omega$. A une fréquence donnée, la longueur d'onde doit avoir une valeur minimale.
	
	 D'autre part, la condition $f(y=b)=0$ implique que $\sin(Kb)=0$, cad que $Kb=m\pi$, où $m$ est un entier, cad :
	 \begin{align*}
	 	\frac{\omega^2}{c^2}-k^2=\frac{m^2\pi^2}{b^2}
	 \end{align*}
	 
	On a alors :	 
	\begin{align*}
		\vec{E}=E_0\sin\left(\frac{m\pi}{b}y\right) \exp[i(kz-\omega t)]\vec{e}_x
	\end{align*}
		 
	 On parle de mode de quantification car pour une pulsation donnée, seuls des valeurs du vecteur d'onde $k$ sont autorisées, discrétisées par la valeur $m$. Pour $m=0$, il n'y a  aucun mode, donc pas de propagation possible ($f=0$). Pour $m=1$, on a un mode, cad un $k$ pour un $\omega$ donné, compris entre $[c\pi/b;2c\pi/b]$. Pour $m=2$, on a deux modes, cad 2 $k$ pour un $\omega$ donné, compris entre $[2c\pi/b;3c\pi/b]$. Etc. Inversement, pour une pulsation $\omega$ donnée, on a $E(\omega b/c\pi)$ modes possibles.
	 
	 Pour représenter $\vec{E}$, il suffit de représenter les fonctions $\sin(\frac{\pi}{b}y)$ (m=1) et $\sin(\frac{2\pi}{b}y)$ (m=2).
	
	\item[$\bigstar$] Le champ magnétique $\vec{B}$ est donné par Maxwell-Faraday :
	\begin{align*}
		\nabla\wedge \vec{E}=i\omega\cdot\vec{B}
	\end{align*}
	On doit trouver :
	\begin{align*}
		\vec{B}&=\frac{k}{\omega}E_0\sin\left(\frac{m\pi}{b}y\right) \exp[i(kz-\omega t)]\vec{e}_y \\
		&+\frac{im\pi}{\omega b}E_0\cos\left(\frac{m\pi}{b}y\right) \exp[i(kz-\omega t)]\vec{e}_z
	\end{align*}
	
	\item[$\bigstar$] Le vecteur $\vec{\Pi}$ est défini comme :
	\begin{align*}
		\vec{\Pi}=\frac{\vec{E}\wedge\vec{B}}{\mu_0}
	\end{align*}
	NE JAMAIS OUBLIER DE PASSER E ET B EN REEL ! 
	
	On trouve, après calculs :
	\begin{align*}
		\vec{\Pi}&=\frac{m\pi}{\omega b\mu_0}E_0^2\sin\left(\frac{m\pi}{b}y\right)\cos\left(\frac{m\pi y}{b}\right)\cos(kz-\omega t)\sin(kz-\omega t)\cdot\vec{e}_y \\
		&+\frac{k}{\omega\mu_0}E_0^2\sin^2\left(\frac{m\pi y}{b}\right)\cos^2(kz-\omega t)\cdot\vec{e}_z
	\end{align*}
	
	Seule la valeur moyenne temporelle de la composante selon 	$\vec{e}_z$ est non nulle : l'énergie se propage bien selon la direction de propagation de l'onde.
	
\end{itemize}

\section*{Réflexion sur un conducteur de conductivité finie}

\begin{itemize}
	
	\item[$\clubsuit$] Equation de propagation d'une onde, ou d'Alembert. En injectant l'expression des ondes incidentes et réfléchies, on trouve que $k_i^2=k_r^2=\omega^2/c^2$. Le signe est donné par la direction de propagation : on a donc $k_i=\omega/c$ et $k_r=-\omega/c$.
	
	\item[$\clubsuit$] Pour des fréquences inférieures au GHz, on peut négliger dans l'équation de Maxwell-Ampère le courant de déplacement $\vec{j}_D=\varepsilon_0\frac{\partial \vec{E}}{\partial t}$ les courants induits par ce même champ électrique, $\vec{j}=\gamma\vec{E}$, car $\omega\varepsilon_0\ll\gamma$. On en déduit que le champ électrique vérifie une équation de diffusion :
	\begin{align*}
		\Delta \vec{E}=\mu_0\gamma\frac{\partial \vec{E}}{\partial t}
	\end{align*}
	En insérant l'expression proposée sur l'onde transmise, on trouve que $k_t^2=-j\mu_0\gamma\omega$. On a donc :
	\begin{align*}
		k_t = \frac{1-j}{\delta}
	\end{align*}
	avec $\delta=\sqrt{2/\mu_0\gamma\omega}$.
	
	\item[$\clubsuit$] On trouve à partir de la relation de structure $\vec{B}=\frac{\vec{k}\wedge\vec{E}}{\omega}$ :
	\begin{align}
    \begin{cases}
        \vec{B}_i=\vec{B_{0,i}}\exp[i(\omega t-kz)]\vec{e}_y\\
     	\vec{B}_r=\vec{B_{0,r}}\exp[i(\omega t+kz)]\vec{e}_y\\
     	\vec{B}_t=\vec{B_{0,t}}\exp[-z/\delta]\exp[i(\omega t-z/\delta)]\vec{e}_y
    \end{cases}  
	\end{align}
	avec $k=\omega/c$, $\vec{B_{0,i}}=\frac{\vec{e}_z\wedge E_{0,i}}{c}$, $\vec{B_{0,r}}=\frac{\vec{e}_z\wedge E_{0,r}}{c}$ et $\vec{B_{0,t}}=\frac{1-i}{\delta}\frac{\vec{e}_z\wedge E_{0,t}}{\omega}$
	
	\item[$\clubsuit$] Il y a continuité des champs électriques et magnétiques : il n'y a ni charges surfaciques, ni courants surfaciques (la conductivité est finie !). On a alors :
	\begin{align}
    \begin{cases}
        E_{0,i} + E_{0,r} =E_{0,t}\\
     	\frac{\vec{E}_{0,i}}{c} - \frac{\vec{E}_{0,r}}{c} =\vec{E}_{0,t}\frac{(1-j)}{\delta\omega}
    \end{cases}  
	\end{align}
	On trouve alors en faisant le produit vectoriel de la seconde ligne avec $\vec{e}_z$ (on se débarrase des vecteurs) :
	\begin{align}
    \begin{cases}
        r=\frac{1-n}{1+n}=\frac{1-\frac{(1-j)c}{\omega\delta}}{1+\frac{(1-j)c}{\omega\delta}}\\
     	t=\frac{2}{1+n}=\frac{2}{1-\frac{(1-j)c}{\omega\delta}}
    \end{cases}  
	\end{align}
	avec $n=\frac{(1-j)c}{\omega\delta}$. 
		
	\item[$\clubsuit$] On bourrine avec $\vec{\Pi}=\frac{\vec{E}\wedge\vec{B}}{\mu_0}$, toujours en notation réelle ! On peut utiliser $(1-j)/\sqrt{2}=\exp(-j\pi/4)$.
	On trouve alors :
	\begin{align}
    \begin{cases}
        \left\langle \vec{\Pi}_i\right\rangle =\frac{E_{0,i}^2}{2\mu_0c}\vec{e}_z\\
     	\left\langle \vec{\Pi}_r\right\rangle=\frac{E_{0,r}^2}{2\mu_0c}\vec{e}_z\\
     	\left\langle \vec{\Pi}_t\right\rangle= \frac{E_{0,t}^2}{2\mu_0\omega\delta}\vec{e}_z
     \end{cases}  
	\end{align}
	
	En $z=0$, on a :
	\begin{align}
    \begin{cases}
        R=-\frac{\left\langle \vec{\Pi}_r\right\rangle\cdot\vec{e}_z}{\left\langle \vec{\Pi}_i\right\rangle\cdot\vec{e}_z}=\frac{1+\left( 1-\frac{\omega\delta}{c}\right)^2}{1+\left(1+\frac{\omega\delta}{c}\right)^2}\\
     	T=-\frac{\left\langle \vec{\Pi}_t\right\rangle\cdot\vec{e}_z}{\left\langle \vec{\Pi}_i\right\rangle\cdot\vec{e}_z}=\frac{4\frac{\omega\delta}{c}}{1+\left(1+\frac{\omega\delta}{c}\right)^2}
    \end{cases}  
	\end{align}	
	On trouve évidemment que $R+T=1$ : l'énergie est belle et bien conservée. 
	
	\item[$\clubsuit$] La puissance dissipe par unité de volume par effet Joule dans le métal s'écrit :
	\begin{align*}
		p_{vol}=\vec{j}\cdot\vec{E_t}=\gamma\vec{E_t}\cdot\vec{E_t}
	\end{align*}
	Pour obtenir une puissance dissipée, on intègre sur un cylindre de section $S$, sur la longueur $z\in[0;\infty]$ pour prendre en compte la dissipation sur toute l'épaisseur du métal. La décroissance étant exponentielle, l'intégrale sera définie :
	\begin{align*}
		\left\langle P \right\rangle &= S\gamma\frac{E_{0,t}^2}{2}\int_0^\infty dz\exp[-2z/\delta] \\
		&= S\gamma\delta|t|^2\frac{E_{0,i}^2}{2} \\
		&= ST\frac{E_{0,i}^2}{2\mu_0 c} \\
		&=S\left\langle \parallel\vec{\Pi}_t(z=0)\parallel\right\rangle
	\end{align*}
	
\end{itemize}

\end{document}