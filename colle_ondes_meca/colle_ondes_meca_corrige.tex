 \documentclass{report}
 
\usepackage[utf8]{inputenc} 
\usepackage[T1]{fontenc}      
\usepackage[top=2.0cm, bottom=3cm, left=3.0cm, right=3.0cm]{geometry}
\usepackage{graphicx}
\usepackage{wrapfig}
\usepackage{amsmath,esint }
\usepackage{amssymb}
\graphicspath{{figures/}{../figures}}

\newcommand*\dif{\mathop{}\!\mathrm{d}}
\newcommand*\diver{\mathop{}\!\mathrm{div}}
\newcommand*\grad{\mathop{}\!\mathrm{grad}}

\begin{document}

\section*{Chaîne suspendue}

\subsubsection{Cas statique}

\begin{itemize}

	\item[$\star$] On prend un élément infinitésimal de corde de longueur $dl=\sqrt{dx^2+dy^2}$. On note $\alpha(x)$ l'angle de la corde par rapport à l'horizontal à l'abscisse $x$. On applique le principe fondamental de la statique :
	\begin{align*}
	\left\lbrace
	\begin{array}{ccc}
	-T(x)\cdot\cos(\alpha(x))+T(x+dx)\cdot\cos(\alpha(x+dx))=0\\
	\\
	-T(x)\cdot\sin(\alpha(x))+T(x+dx)\cdot\sin(\alpha(x+dx))-\mu g\sqrt{dx^2+dy^2}=0\\
	\end{array}\right.
	\end{align*}		
	
	De la première équation, on voit que $T(x)\cdot\cos(\alpha(x))=cste=T_0\cos(\alpha_0)$, où $T_0$ et $\alpha_0$ sont la tension et l'angle au début de la corde (par exemple. On a donc $T(x)=T_0\cos(\alpha_0)/\cos(\alpha(x))$.
	
	La seconde équation s'écrit :
	\begin{align*}
		\frac{\dif }{\dif x}T(x)\cdot\sin(\alpha(x))=\mu g \sqrt{dx^2+dy^2}
	\end{align*}
	Avec la relation trouvée sur la tension, on obtient :
	\begin{align*}
		dx\frac{\dif }{\dif x}\left[T_0\cos(\alpha_0)\tan(\alpha(x))\right] =\mu g dx\sqrt{1+\frac{dy^2}{dx^2}}
	\end{align*}
	Comme $\tan(x)=dy/dx$, on tombe sur l'équation différentielle :
	\begin{align*}
		\frac{\dif^2 y}{\dif x^2} =\frac{1}{l_c}\sqrt{1+\left( \frac{dy}{dx}\right)^2}
	\end{align*}	
	avec $l_c=T_0\cos(\alpha_0)/\mu g$.
	\item[$\star$] Avec le changement de variable $p(x)=\dif y/\dif x$, on a :
	\begin{align*}
		\frac{\dif p}{\dif x} =\frac{1}{l_c}\sqrt{1+p(x)^2}
	\end{align*}	
	On obtient alors :
	\begin{align*}
		\frac{\dif p}{\sqrt{1+p(x)^2}} =\frac{\dif x}{l_c}
	\end{align*}	
	On reconnait que la primitive est la fonction inverse du sinus hyperbolique :
	\begin{equation}
		\sinh^{-1}(p)=\frac{x}{l_c} +\alpha
	\end{equation}
	On obtient alors :
	\begin{equation}
		y(x) = l_c\cosh\left( \frac{x}{l_c}+\alpha\right) +\beta
	\end{equation}
	
	Avec les conditions aux limites ($y(-D/2)=y(+D/2)=0$), on a :
	\begin{align*}
	y(x)=l_c\left[\cosh\left(\frac{x}{l_c} \right) -\cosh\left(\frac{D}{2l_c} \right) \right] 
	\end{align*}
	\item[$\star$] La tension horizontale est constante et vaut $T_h(x)=T_0\cos(\alpha_0)$. La tension verticale est $T_v(x)=T(x)\sin(\alpha(x))=T_0\cos(\alpha_0)\tan(\alpha(x))=T_0\cos(\alpha_0)\frac{dy}{dx}$. On a donc :
	\begin{align*}
		T_v(x)=T_0\cos(\alpha_0)\sinh\left(\frac{x}{l_c} \right) 
	\end{align*}
	
	\item[$\star$] La longueur correspond à l'intégrale curviligne :
	\begin{align*}
		L=\int_C \dif l=\int_{-D/2}^{D/2}dx\sqrt{1+\left( \frac{dy}{dx}\right)^2}
	\end{align*}
	En utilisant l'équation différentielle trouvée précédemment, on a tout simplement :
	\begin{align*}
		L=\int_{-D/2}^{D/2}dx\frac{d^2y}{dx^2}=\left[ \frac{dy}{dx}\right]_{-D/2}^{D/2}=2l_c\sinh\left(\frac{D}{2l_c} \right) 
	\end{align*}
	La flèche correspond tout simplement à la différence entre le point le plus haut et le plus bas, soit $-y(0)$ :
	\begin{align*}
		h=l_c\left[\cosh\left(\frac{D}{2l_c} \right) -1 \right] 
	\end{align*}
	On utilise la relation $\cosh^2-\sinh^2=1$ :
	\begin{align*}
		\left( \frac{h}{l_c}+1\right) ^2-\left( \frac{L}{2l_c}\right) ^2=1
	\end{align*}
	et donc :
	\begin{align*}
		l_c=\frac{L^2/4-h^2}{2h}
	\end{align*}
	Ainsi, avec simplement une photo d'une chaîne suspendue, on peut connaitre $L$, $h$ et $\alpha_0$, on en déduit $l_c$ qui nous donne l'information sur $T_0$
\end{itemize}

\subsubsection*{Cas dynamique}

\begin{itemize}

	\item[$\diamond$] A ce moment là $T_0\gg \mu g$, et donc $l_c\longrightarrow\infty$ et la corde est horizontale. L'angle $\alpha(x)$ est très petit. On néglige la gravité dans ce cas-là.

	\item[$\diamond$] On reprend le même raisonnement que précédemment en appliquant le principe fondamental de la dynamique et en négligeant la pesanteur. On trouve une équation d'Alembert, qui correspond à la propagation des ondes dans la corde :
	\begin{align*}
		\frac{\partial^2y}{\partial x^2}=\frac{1}{c^2}\frac{\partial^2y}{\partial t^2}
	\end{align*}
	avec $c=T_0/\mu$.
	Les solutions sont de la forme $y(x,t)=f(t-x/c)+g(t+x/c)$ : cela correspond à des ondes se propageant suivants les $x$ croissants ($f$) et les $x$ décroissants ($g$).
	
	\item[$\diamond$] Dans le cas d'ondes stationnaires, on a $y(x,t)=F(x)G(t)$, cad indépendance entre les variables de temps et d'espace. En injectant dans l'équation de propagation, on trouve que la fonction $F$ est une fonction sinusoïdale, qui avec les conditions aux limites est nécessairement :
	\begin{align*}
		F(x)=F_0\sin(k_n x),\quad k_n=\frac{n\pi}{L}
	\end{align*}
	Et d'autre part :	
	\begin{align*}
		G(t)=G_1\cos(\omega t)+G_2\sin(\omega t)
	\end{align*}	
En réinjectant les solutions de $F(x)G(t)$ trouvées, on tombe sur la relation de dispersion :
	\begin{align*}
		\omega=\omega_n=k_nc=\frac{n\pi c}{L}
	\end{align*}	
	
	Comme toute superposition des solutions précédentes au mode $n$ vérifient l'équation de propagation, la solution générale est donc une somme des solutions au mode $n$ :
	\begin{align*}
		y(x,t)=\sum_n \left[A_n\cos(\omega_nt) + B_n\sin(\omega_nt)\right] \cdot\sin(k_nx)
	\end{align*}
	\item[$\diamond$] On commence par relier les coefficients $A_n$ et $B_n$ avec les conditions initiales :
	\begin{align*}
	\left\lbrace
	\begin{array}{ccc}
	y(x,0)=\sum_n A_n\sin(k_nx)\\
	\\
	\frac{\dif y}{\dif t}(x,0)=\sum_n \omega_n B_n\sin(k_nx)\\
	\end{array}\right.
	\end{align*}		
	On peut inverser les intégrales en utilisant l'orthogonalité des fonctions sinusoïdales :
	\begin{align*}
	\int_0^1\dif u\sin(n\pi u)\sin(m\pi u)=\frac{\delta_{nm}}{2}
	\end{align*}			
On a alors :
	\begin{align*}
	\left\lbrace
	\begin{array}{ccc}
	A_n=\frac{2}{L}\int_0^L\dif x\cdot y(x,0)\sin\left(\frac{n\pi x}{L} \right) \\
	\\
	B_n=\frac{2}{n\pi c}\int_0^L\dif x\cdot \frac{\dif y}{\dif t}(x,0)\sin\left(\frac{n\pi x}{L} \right)\\
	\end{array}\right.
	\end{align*}		
	
	Avec les conditions initiales données, on a $\frac{\dif y}{\dif t}(x,0)=0$, cad $B_n=0$. De la même manière :
	\begin{align*}
		A_n=\frac{2}{L}\int_0^a\dif x \frac{h}{a}x\sin\left(\frac{n\pi x}{L} \right) + \frac{2}{L}\int_a^L\dif x \frac{h(L-x)}{L-a}\sin\left(\frac{n\pi x}{L} \right) 
	\end{align*}
	En intégrant par partie, on obtient :
	\begin{align*}
		A_n=\frac{2h}{n^2\pi^2}\frac{L^2}{a(L-a)}\sin\left(\frac{n\pi a}{L} \right) 
	\end{align*}
	\item[$\diamond$] Ici, les harmoniques décroissent en $1/n^2$. En fonction des conditions initiales, on peut avoir d'autres décroissances, qui dépendent du type d'instrument de musique. Cela détermine alors le nombre d'harmoniques qui détermine le timbre de l'instrument. 
	
\end{itemize}

\section*{Corde pendue verticalement}

\begin{itemize}

	\item[$\ast$] En appliquant le principe fondamental de la dynamique, on trouve :
	\begin{align*}
	\left\lbrace
	\begin{array}{ccc}
	T(z+dz)\cdot\cos(\alpha(z+dz))-T(z)\cdot\cos(\alpha(z))+\mu g\dif z=0\\
	\\
	\mu\dif z\frac{\partial^2\Psi}{\partial t^2}=T(z+dz)\cdot\sin(\alpha(z+dz))-T(z)\cdot\sin(\alpha(z))\\
	\end{array}\right.
	\end{align*}		
	Si on prend l'hypothèse $\sin(\alpha(z))\simeq\tan(\alpha(z))\simeq\alpha(z)\simeq\frac{\partial \Psi}{\partial z}$ et $\cos(\alpha(z))\simeq1$, on obtient :
	\begin{align*}
	\left\lbrace
	\begin{array}{ccc}
	T(z)=\mu gz\\
	\\
	\mu\frac{\partial^2\Psi}{\partial t^2}=\frac{\partial}{\partial z}(T(z)\alpha(z))\\
	\end{array}\right.
	\end{align*}		
	En rentrant l'expression de la tension, on obtient donc :
	\begin{align*}
		\frac{\partial^2}{\partial t^2}\Psi(z,t)=g\frac{\partial}{\partial z}\left( z\frac{\partial}{\partial z}\Psi(z,t)\right) 
	\end{align*}
	
		\item[$\ast$] Ce sont des solutions stationnaires. En injectant, on trouve :
		\begin{align*}
			\left[ \omega^2\alpha(z)+g\frac{\partial}{\partial z}\left( z\frac{\partial}{\partial z}\alpha(z,t)\right)\right] \cos(\omega t)+\left[ \omega^2\beta(z)+g\frac{\partial}{\partial z}\left( z\frac{\partial}{\partial z}\beta(z,t)\right)\right] \sin(\omega t)=0
		\end{align*}
		Comme cette équation est vraie $\forall t$, on a nécessairement :
trouve :
		\begin{align*}
			\omega^2\alpha(z)+g\frac{\partial}{\partial z}\left( z\frac{\partial}{\partial z}\alpha(z,t)\right)=0
		\end{align*}		
		
		$\alpha$ et $\beta$ vérifient la même équation différentielle en $z$.
		
		\item[$\ast$] Avec le changement de variable, on a $\alpha(z)=\alpha(0)A(Z(z))$, $\alpha'(z)=\alpha(0)\frac{\omega^2}{g}A'(Z(z))$ et $\alpha''(z)=\alpha(0)\frac{\omega^4}{g^2}A''(Z(z))$. L'équation différentielle devient donc :
		\begin{equation}
			A(z)+A'(Z)+ZA''(Z)=0
		\end{equation}
		
		\item[$\ast$] La série entière s'écrit :
		\begin{align*}
			A(Z) = 1+\sum_{k=1} A_kZ^k
		\end{align*}
		Injecté dans l'équation différentielle, on trouve :
	\begin{align*}
		1+A_1+\sum_{k=1}(A_k+(k+1)A_{k+1}+k(k+1)A_{k+2})Z^k=0
	\end{align*}		
	On en déduit $A_1=-1$ et $A_{k+1}=-A_k/(k+1)^2$, donc :
	\begin{align*}
		A_k=\frac{(-1)^2}{(k!)^2}
	\end{align*}		
		
		\item[$\ast$] Comment pourrait-on trouver une relation de dispersion $\omega(k)$ ?
		
\end{itemize}

\section*{Propagation sur une ligne électrique}

\begin{itemize}

	\item[$\spadesuit$] Une loi des noeuds et une loi des mailles donnent : 
	\begin{align*}
	\left\lbrace
	\begin{array}{ccc}
	V_{n-1}-V_{n}=L\frac{\dif I_{n}}{\dif t}\\
	\\
	I_{n}=I_{n+1}+C\frac{\dif V_{n}}{\dif t}\\
	\\
	V_{n}-V_{n+1}=L\frac{\dif I_{n+1}}{\dif t}\\
	\end{array}\right.
	\end{align*}		
	Avec ces trois expressions, on obtient facilement :
	\begin{equation}
		\frac{\dif^2 V_n}{\dif t^2} =\omega_0^2(V_{n+1}+V_{n-1}-2V_n)
		\label{eq:prop}
	\end{equation}
	avec $\omega_0=1/\sqrt{LC}$.
	 \item[$\spadesuit$] Avec les relations précédentes, on obtient : 
	 \begin{align*}
	 \frac{\dif}{\dif t}\left( \frac{1}{2}CV_n^2 + LI_n^2 \right) =V_n(I_{n}-I_{n+1})- I_n(V_{n}-V_{n-1})=I_nV_{n-1}-I_{n+1}V_n
	 \end{align*}
	 Le terme global représente la variation temporelle de l'éerngie contenue dans une cellule $n$, $I_nV_{n-1}$ est la puissance qui est reçue depuis la cellule $n-1$ et le terme $I_{n+1}V_n$ est la puissance qui est transmise à la cellule $n+1$. C'est un bilan d'énergie.
	 
	\item[$\spadesuit$] C'est un retard exprimé en déphasage qui s'ajoute à chaque traversée d'une cellule. Avec une récurrence : $A_n=A_0\exp(-jn\alpha)$. On injecte cette expression dans l'équation de "propagation", et on trouve :
	\begin{align*}
	\left\lbrace
	\begin{array}{ccc}
	\omega^2&=&2\omega_0^2(1-\cos(\alpha)) \\
	\\
	& =  &4\omega_0^2\sin^2\left(\frac{\alpha}{2}\right) 
	\end{array}\right.
	\end{align*}	

	\item[$\spadesuit$] Comme le sinus est borné, on a forcément $\omega<\omega_c=2\omega_0$. La phase de la tension est donc $\omega t -n\alpha$, ce qui correspond à une propagtion de cellule en cellule. La vitesse de propagation est $v_\varphi=\omega/\alpha$, qui correspond au nombre de cellules parcourues par unité de temps (ce n'est pas en m/s mais en s$^{-1}$ !).
	
	\item[$\spadesuit$] Si $\omega\ll\omega_c$, alors $\sin(\alpha/2)\simeq\alpha/2$ et donc $\omega\simeq\omega_0\alpha$ et donc $v_\varphi=\omega_0$. La vitesse de phase ne dépend pas de $\omega$ donc il n'y a pas de dispersion. Le retard est donc $\tau=1/\omega_0$ Application numérique : on trouve $\omega_0=2\cdot10^6$rad/s et $\tau=5\cdot10^{-7}$s. Il faut donc 200 cellules. 

	\item[$\spadesuit$] La vitesse de groupe est $v_g=\dif \omega /\dif \alpha$, elle correspond à la propagation de l'information d'un paquet d'onde (ou de l'énergie). On a $v_g=\omega_0\cos(\alpha/2)$. Pour $\alpha=\pi$, $v_g=0$, il n'y a plus de propagation. Cela correspond à la pulsation de coupure $\omega_c$.
	
	\item[$\spadesuit$] En utilisant la relation entre le courant $I_n$ et les tensions aux bornes de la cellule, on obtient :
\begin{align*}
	B_n=\frac{A_n}{jL\omega}(\exp(j\alpha)-1)=\frac{2A_n}{L\omega}\exp(j\alpha/2)\sin(\alpha/2)=\frac{A_n}{L\omega_0}\exp(j\alpha/2)
\end{align*}	
	 Pour calculer la moyenne temporelle de l'énergie de la cellule, il faut nécessairement passer en réel (sous peine d'avoir des valeurs moyennes nulles !) : $V_n=\Re[A_n\exp(j\omega t)]$ et $I_n=\Re[B_n\exp(j\omega t)]$. Alors :
	   	\begin{align*}
	   		\left\langle\frac{1}{2}CV_n^2 \right\rangle =\frac{1}{4}C\mid A_n\mid^2
	   	\end{align*}
	    \begin{align*}
	   		\left\langle\frac{1}{2}LI_n^2 \right\rangle =\frac{1}{4}\frac{L}{L^2\omega_0^2}\mid A_n\mid^2=\frac{1}{4}C\mid A_n\mid^2
	   	\end{align*}
	   	
	   	On obtient donc :
	   	\begin{align*}
	   		E=\frac{1}{2}C\mid A_n\mid^2
	   	\end{align*}
	   	La cellule reçue de la cellule $n-1$ s'écrit :
	   	\begin{align*}
	   		P=\left\langle V_{n-1}(t)I_n(t) \right\rangle =\frac{1}{2}C\mid A_n\mid^2\frac{\cos(\alpha/2)}{LC\omega_0}
	   	\end{align*}
	   	Comme $\omega_0\cos(\alpha/2)=v_g$, on a alors :
	   	\begin{align*}
	   		P=\frac{1}{2}C\mid A_n\mid^2v_g
	   	\end{align*}
	   	On obtient donc :
	   	\begin{align*}
	   		\frac{P}{E}=v_g
	   	\end{align*}
	   	La vitesse de groupe est donc la vitesse de propagation de l'énergie. 

\end{itemize}

\section*{Propagation dans une ligne coaxiale dissipative}

\begin{itemize}

	\item[$\spadesuit$] 	
	Une loi des noeuds et une loi des mailles donnent : 
	\begin{align*}
	\left\lbrace
	\begin{array}{ccc}
	V_{n-1}-V_{n}=L\frac{\dif I_{n}}{\dif t} + RI_n\\
	\\
	I_{n}=I_{n+1}+C\frac{\dif V_{n}}{\dif t}+GV_n\\
	\end{array}\right.
	\end{align*}		
	Avec ces trois expressions, on obtient facilement :
	\begin{equation}
		LC\frac{\dif^2 V_n}{\dif t^2} + (LG+RC)\frac{\dif V_{n}}{\dif t}+RGV_n=V_{n+1}+V_{n-1}-2V_n
		\label{eq:prop}
	\end{equation}
	
	 \item[$\spadesuit$] 
	 \begin{align*}
	 \frac{\dif}{\dif t}\left( \frac{1}{2}CV_n^2 +\frac{1}{2} LI_n^2 \right) &=V_n(I_{n}-I_{n+1}-GV_n)+ I_n(V_{n-1}-V_{n}-RI_n) \\
	 &=I_nV_{n-1}-I_{n+1}V_n-GV_n^2-RI_n^2
	 \end{align*}
	 Le terme global représente la variation temporelle de l'éerngie contenue dans une cellule $n$, $I_nV_{n-1}$ est la puissance qui est reçue depuis la cellule $n-1$ et le terme $I_{n+1}V_n$ est la puissance qui est transmise à la cellule $n+1$. Les termes $GV_n^2$ et $RI_n^2$ sont els pertes dues à la conductance et à la résistance. C'est un bilan d'énergie.

	\item[$\spadesuit$] On a tout simplement $C = ca$, $G=ga$, $L=la$ et $R=ra$. On peut aussi montrer que ce sont des grandeurs linéiques infinétisémiales comme $a=dx$.

	\item[$\spadesuit$] Comme $V_n(t)=V(x,t)=V(na,t)$ et que $a=dx$, on a alors $V_{n+1}(t)=V((n+1)a,t) = V(na+a,t)=V(x+dx,t)$.
	On peut montrer que $V_{n+1}+V_{n-1}-2V_n=dx^2\frac{\partial^2V}{\partial x^2}$ avec un d.l. à l'ordre 2 en $dx$. Alors :
	\begin{equation}
		lc\frac{\partial^2 V}{\partial t^2} + (lg+rc)\frac{\partial V}{\partial t}+rgV=\frac{\partial^2V}{\partial x^2}
		\label{eq:prop}
	\end{equation}

	\item[$\spadesuit$] On retombe sur l'équation d'Alembert. Les solutions sont de la forme :
	\begin{align*}
		V(x,t)=f\left(t-\frac{x}{c_0} \right) + g\left(t+\frac{x}{c_0} \right) 
	\end{align*}
	La vitesse de propagation est bien $c_0^2=1/lc$.
	\item[$\spadesuit$] Il suffit d'injecter l'expression proposée dans l'équa. diff. :
	\begin{equation}
		k^2=lc\omega^2-j(lg+rc)-rg
	\end{equation}

 \item[$\spadesuit$] En factorisant par $lc$, on obtient :
 \begin{align*}
 	k^2=\frac{\omega^2}{c_0^2}\left(1 - \frac{j}{\omega}\left( \frac{r}{l}+\frac{g}{c}\right)  - \frac{1}{\omega^2}\frac{rg}{lc}\right) 
 \end{align*}

Comme $(1+x)^\frac{1}{2}=1+x/2-x^2/8$, en ne gardant que les termes d'ordre 2, on obtient : 
\begin{align*}
	k=\frac{\omega}{c_0}\left[1-\frac{j}{2\omega}\left( \frac{r}{l}+\frac{g}{c}\right)+\frac{1}{8\omega^2}\left( \frac{r}{l}-\frac{g}{c}\right)^2 \right] 
\end{align*}

La tension s'écrit alors : $V(x,t)=V_0\exp\left[j(\omega t - k'x )\right]\exp\left[-k"x \right]$. On a alors une longueur caractéristique de décroissance $\delta=1/k''$. Pas de dispersion si $v_\varphi$ ne dépend pas de $\omega$, cad $k'$ ne dépend pas de $\omega$, cad $\frac{r}{l}-\frac{g}{c}=0$.

\end{itemize}

\end{document}
