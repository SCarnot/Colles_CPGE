 \documentclass{report}
 
\usepackage[utf8]{inputenc} 
\usepackage[T1]{fontenc}      
\usepackage[top=2.0cm, bottom=3cm, left=3.0cm, right=3.0cm]{geometry}
\usepackage{graphicx}
\usepackage{wrapfig}
\usepackage{amsmath,esint }
\graphicspath{{figures/}{../figures}}

\newcommand*\dif{\mathop{}\!\mathrm{d}}
\newcommand*\diver{\mathop{}\!\mathrm{div}}
\newcommand*\grad{\mathop{}\!\mathrm{grad}}

\begin{document}

\section*{Exercice 1}

\subsubsection*{Approche en mécanique classique}

\begin{itemize}
	\item[$\clubsuit$] Dans une volume $a\Sigma$, on a une charge $+e$ et une charge $-e$, comme celle-ci sont réparties uniformément. Les densités de charges sont donc respectivement $\rho_+=e/(a\Sigma)$ et $\rho_-=-e/(a\Sigma)$.
	\item[$\clubsuit$] Par définition, $\vec{j}=\rho\vec{v}$. Comme seuls les électrons ont une vitesse non nulle, $\vec{j}=-v'\rho_-=-ev'/(a\Sigma)$. Et donc $I=\oiint_\Sigma\dif\vec{S}\vec{j}=-ev'/a$.
	\item[$\clubsuit$] $\rho_{tot}=\rho_++\rho_-=0$ donc $\vec{E}=0$.
	\item[$\clubsuit$] Avec le théorème d'Ampère appliqué uniquement en dehors du fil, on trouve :
	\begin{align*}
		\vec{B}=-\frac{\mu_0I}{2\pi r}\vec{e_\theta}
	\end{align*}
	(le signe - provient du fait que le sens du courant est opposé à celui des électrons)
	La force qui s'exerce sur la charge $q$ est donc :
	\begin{align*}
		\vec{F}=-\frac{qv\mu_0I}{2\pi r}\vec{e_r}
	\end{align*}
\end{itemize}

\subsubsection*{Approche en mécanique relativiste}

	\begin{itemize}
	
		\item[$\clubsuit$] Ainsi, en se déplaçant à la vitesse $\vec{v}$, la charge $+q$ voit dans son référentiel la distance entre atomes réduite d'un facteur $\gamma_{+}=1/\sqrt{1-v^{2}/c^{2}}$ et la distance entre électrons de conduction d'un facteur $\gamma_{-}=1/\sqrt{1-(v-v')^{2}/c^{2}}$. 
		On trouve donc que :
		\begin{align*}
			\rho_+=\frac{e}{a\Sigma}\frac{1}{\sqrt{1-v^{2}/c^{2}}}
		\end{align*}
		\begin{align*}
			\rho_-=\frac{e}{a\Sigma}\frac{1}{\sqrt{1-(v-v')^{2}/c^{2}}}
		\end{align*}		
		Attention, l'hypothèse que la vitesse relative des électrons par rapport à la charge $q$ est $v'-v$ est une approximation. En mécanique relativiste, la vitesse relative serait :
		\begin{equation}
			v_{e_-/q}=\frac{v'-v}{1-\frac{vv'}{c^2}}
		\end{equation}
		
	\item[$\clubsuit$] On trouve facilement avec le théorème dee Gauss que :
	\begin{align*}
		\vec{E}=\frac{\Sigma(\rho_++\rho_-)}{2\pi r\epsilon}
	\end{align*}
	
	\item[$\clubsuit$] En développant à l'ordre 2, on trouve :
	\begin{align*}
		\rho_++\rho_-\approx \frac{e}{a\Sigma}\frac{vv'}{c^2}
	\end{align*}
	On trouve alors que :
	\begin{align*}
		\vec{E}=-\frac{I\mu_0 v}{2\pi r}\vec{e_r}
	\end{align*}
	La force de Lorentz associée est donc :
	\begin{align*}
		\vec{F}=-\frac{qI\mu_0 v}{2\pi r}\vec{e_r}
	\end{align*}
	
	Cette expression est identique à celle trouvée par le calcul du champ magnétique en mécanique classique. Le champ magnétique est-il est une approximation à l'ordre 2 de la force de Coulomb ? 
	\end{itemize}

\section*{Exercice 3}

\begin{itemize}
	
	\item[$\spadesuit$] On raisonne sur un ensemble d'électrons. On considère les évènements ayant eu lieu à partir de $t=0$. Il faut calculer d'abord le nombre d'électrons ayant subi une collision entre $t$ et $t+dt$. Ce nombre est $N(t)-N(t+dt)=N(t)/\tau$. On a donc $N(t)=N_0\exp(-t/\tau)$. Pour un électron donné, la probabilité de ne pas subir de collision est donc $P(t)=N(t)/N_0$. 
	
	\item[$\spadesuit$] Soit $N_0$ le nombre total d'électrons. Entre $t$ et $t+dt$, il y a eu $dtN_0/\tau$ qui ont subi une collision. La quantité de mouvement de tous ces électrons est donc perdue. D'autre part, entre $t$ et $t+dt$ chaque électron est soumis à la force $\vec{F}(t)$, faisant changer la quantité de mouvement totale de $N_0F(t)dt$. Finalement, il vient :
	\begin{equation}
		\vec{P}(t+dt)=\vec{P}(t)-\frac{dt}{\tau}\vec{P}(t)+N_0\vec{F}dt
	\end{equation}
	
	\item[$\spadesuit$] On en déduit la vitesse moyenne d'un électron, définie par $\vec{v}(t)=\vec{P}/(mN_0)$ :
	\begin{align*}
		\frac{d\vec{v}}{dt}=\frac{\vec{v}}{\gamma}+\vec{F}(t)
	\end{align*}
	où $\gamma=m/\tau$.
	
	\item[$\spadesuit$] La force subie par les électrons est la force de Lorentz : $\vec{F}=-eE_0\exp(-i\omega t)$. L'équation précédente devient :
	\begin{align*}
		-i\omega\vec{v}=-\frac{e}{m}\vec{E_0}-\frac{\vec{v}}{\tau}
	\end{align*}
	On en déduit :
	\begin{align*}
		\vec{v}=\frac{e\tau}{m}\frac{E_0}{i\omega\tau-1}
	\end{align*}
	En introduisant la conductivité $\gamma$, $\vec{j}=\gamma \vec{E}$, où $\vec{j}=-ne\vec{v}$, on trouve :
	\begin{align*}
		\gamma(\omega)=\frac{ne^2\tau}{m}\frac{1}{1-i\omega\tau}
	\end{align*}
	
	\item[$\spadesuit$] Dans un métal, qui est un réseau cristallin, il y a typiquement un électron tous les Angstrom, soit tous les $10^{-10}$m. On obtient des densités typiques de $10^{30}$ atomes par m$^3$.
	
	\item[$\spadesuit$] La résistivité statique correspond à une fréquence qui tend vers 0, cad :
	\begin{align*}
		\rho = \frac{1}{\gamma}=\frac{m}{ne^2\tau}
	\end{align*}
	On trouve donc $\tau\simeq10^{-14}$s.
	
\end{itemize}

\section*{Exercice 4}

\begin{itemize}

	\item[$\heartsuit$] Résistance classique d'un cylindre : $R=L/(\gamma\pi a^2)$.
	
	\item[$\heartsuit$] Isolons le câble arrivant en $A$. Par symétrie, le courant partira dans tous les directions. La densité de courant va s'écrire en un point $M$ :
	\begin{align*}
		\vec{j_A}=\frac{I}{2\pi e }\frac{\vec{e_r}}{\|\vec{AM} \|}
	\end{align*}
	On effectue le même raisonnement pour $B$. La densité de courant totale est alors :
	\begin{align*}
		\vec{j}=\frac{I}{2\pi e }\left[ \frac{\vec{e_r}}{\|\vec{AM} \|} - \frac{\vec{e_r}}{\|\vec{BM} \|}\right] 
	\end{align*}
	
	\item[$\heartsuit$] En intégrant la relation $\vec{j}=\gamma\vec{E}$ le long du chemin $AB$, dont la coordonnée sera donnée par $x$, en faisant varier $x$ de $a$ à $d-a$ (pour éviter une divergence de la densité de courant) :
	\begin{align*}
		\int_a^{d-a}dx j(x)=\frac{I}{2\pi e}\int_a^{d-a}dx\left(\frac{1}{x}+\frac{1}{d-x} \right) =\gamma\int_a^{d-a}dxE=\gamma\int_a^{d-a}dx\frac{dV}{dx}
	\end{align*}
	
	On trouve alors :
	\begin{align*}
		\Delta V = \frac{I}{\pi e\gamma}\log\left(\frac{d-a}{a} \right) 
	\end{align*}

	\item[$\heartsuit$] En se plaçant en coordonnées sphériques, on a :
	\begin{align*}
		\vec{j}=\frac{I}{2\pi}\left[ \frac{\vec{e_r}}{\|\vec{AM} \|^2} - \frac{\vec{e_r}}{\|\vec{BM} \|^2}\right] 
	\end{align*}
	Attention, il y a un facteur 2 par rapport à la surface d'une sphère car il s'agit de demi-sphères.
		On trouve alors :
	\begin{align*}
		\Delta V = \frac{I}{\pi \gamma}\frac{d}{a(d-a)}
	\end{align*}
	
\end{itemize}

\section*{Exercice 5}

\begin{itemize}

	\item[$\diamondsuit$] Les lignes de champs sont radiales cad $\vec{j}=j(r)\vec{e_r}$. On a donc :
\begin{align*}
	\vec{j}(r)=\frac{I}{2\pi r^2}\vec{e_r}
\end{align*}
On en déduit :
\begin{align*}
	\dif V=-E(r)dr=\frac{-I}{2\pi r^2\gamma}dr
\end{align*}
Par intégration, on trouve :
\begin{align*}
	V(r)=\frac{I}{2\pi r\gamma}
\end{align*}

	\item[$\diamondsuit$] Le potentiel de l'hémisphère est donc simplement :
\begin{align*}
	U=\frac{I}{2\pi a\gamma}
\end{align*}
La résistance est donc tout simplement $R=\frac{1}{2\pi a \gamma}$. On trouve qu'elle ne dépasse pas 30$\Omega$ si $a>53$cm.

	\item[$\diamondsuit$] La tension de pas vaut, si $d=1$m :
	\begin{align*}
		V_p(r)=V(r)-V(r+d)=\frac{I}{2\pi \gamma r(r+d)}
	\end{align*}
On trouve que $V_p(10m)=7,2kV$ et  $V_p(100m)=79V$

	\item[$\diamondsuit$] Le courant qui traverse la personne est $i=V_p/R$. On trouve i(10)=2,9A et i(100)=32mA.
\end{itemize}

\section*{Étude d'un colloïde}

\begin{itemize}

	\item[$\heartsuit$] Le milieu est composé de cations et d'anions à la même densité :
	\begin{align*}
		\rho=eN_+-eN_-=-2eN_0\sinh\left( \frac{eV}{k_BT}\right) 
	\end{align*}
Lorsque $eV\ll k_BT$, on a alors : 
\begin{align*}
	\rho\simeq -2N_0\frac{eV}{k_BT}
\end{align*}

	\item[$\heartsuit$] Avec les rotations et les symétries, on montre facilement que $\vec{E}=E(r)\vec{e_r}$. On applique le théorème de Gauss sur un volume compris entre $r$ et $r+dr$ :
	\begin{align*}
		-4\pi r^2E(r)-4\pi(r+\dif r)^2E(r+\dif r)=\frac{4\pi r^2\dif r\rho}{\varepsilon}
	\end{align*}
	On trouve donc :
	\begin{align*}
		\frac{1}{r}\frac{\dif}{\dif r}\left( r^2\frac{\dif V}{\dif r}\right) +\frac{\rho}{\epsilon}=0
	\end{align*}
	Cette éuation correspond à l'équation de Maxwell-Gauss : $\diver\vec{E}=\frac{\rho}{\varepsilon}$ (qui correspond à l'équation de Poisson avec le potentiel).

	\item[$\heartsuit$] On remplace $\rho$ par l'expression trouvée plus haut. On obtient : 
	\begin{align*}
		\frac{\dif^2U}{\dif r^2}-\frac{2N_0e^2}{k_BT\varepsilon}U=0
	\end{align*}
	
	On pose $\lambda^2 = \frac{k_BT\varepsilon}{2N_0e^2}$. C'est une longueur caractéristique de la décroissance du potentiel. Dans de l'eau pure, le pH est égal à 7 donc $N_0=10^{-7}$mol/L$=10^{19}$part.m$^{-3}$, soit $\lambda=1\mu$m.
	
	En résolvant l'équation, on trouve $U(r)=A\exp\left(-r/\lambda \right) +B\exp\left(r/\lambda\right)$. La condition aux limites $V(r\longrightarrow\infty)\longrightarrow0$ impose $B=0$. On a alors :
	\begin{align*}
		V(r)=\frac{A}{r}\exp\left(-\frac{r}{\lambda} \right) 
	\end{align*}
	
	\item[$\heartsuit$] L'expression du champ est :
	\begin{align*}
		\vec{E}=-\frac{\dif V}{\dif r}\vec{e_r}=A\frac{\exp(-r/\lambda)}{r^2}\left(1+\frac{r}{\lambda}\right)\vec{e_r}
	\end{align*}
	S'il n'y avait pas d'ions, on devrait retrouver l'expression du champ d'une particule ponctuelle de charge $Q$. Or l'absence d'ions correspond à $\lambda=\infty$, c'est-à-dire qu'il n'y a plus d'écrantage. On doit nécessairement retrouver $\vec{E}=\frac{Q}{4\pi\epsilon r^2}\vec{e_r}$.
	
	D'autre part, pour $r=r_0$, l'expression du champ \textit{avec} ou \textit{sans} ions autour est la même (car on est collé à la surface de la particule). Donc :
	\begin{align*}
		\frac{Q}{4\pi\epsilon r^2}=A\frac{\exp(-r_0/\lambda)}{r^2}\left(1+\frac{r_0}{\lambda}\right)
	\end{align*}
	
	On a alors :
	\begin{align*}
		V(r)=\frac{Q}{4\pi\epsilon r^2}\frac{1}{1+\frac{r_0}{\lambda}}\exp\left(-\frac{r-r_0}{\lambda} \right) 
	\end{align*}
	
La densité de charge est proportionnelle à l'opposé du potentiel :
	\begin{align*}
		\rho(r)=-\frac{2N_0eQ}{4\pi k_BT\epsilon r^2}\frac{1}{1+\frac{r_0}{\lambda}}\exp\left(-\frac{r-r_0}{\lambda} \right) 
	\end{align*}	
	
\end{itemize}

\section*{Condensateur Terre-ionosphère}

\begin{itemize}

	\item[$\clubsuit$] Les symétries et les invariances donnent $\vec{E}=E(r)\vec{e_r}$. Avec le théorème de Gauss appliqué sur une sphère de rayon $r$, on obtient :
	\begin{align*}
	\left\lbrace
	\begin{array}{ccc}
	r<R & : & \vec{E}=\vec{0} \\
	R<r<R+z_0 & : & \vec{E} = -\frac{Q}{4\pi\varepsilon_0 r^2}\vec{e_r} \\
	r<R+z_0 & : & \vec{E}=\vec{0}
	\end{array}\right.
\end{align*}

	\item[$\clubsuit$] Le potentiel se retrouve grâce à l'équation $\frac{\dif V}{\dif r}=-E(r)$. On a donc :
	\begin{align*}
		V(r) = \frac{Q}{4\pi\varepsilon_0 r}+A
	\end{align*}
Comme le potentiel est nul en $z=0$ (cad en $r=R$) :
	\begin{align*}
		V = V(R+z_0)-V(0) = \frac{Q}{4\pi\varepsilon_0}\left(\frac{1}{R}-\frac{1}{R+z_0} \right) 
	\end{align*}
On trouve une capacité équivalente de :
	\begin{align*}
		C = \frac{4\pi\varepsilon_0R(R+z_0)}{z_0}\simeq\frac{4\pi\varepsilon_0R^2}{z_0}
	\end{align*}
	L'approximation est la formule d'un condensateur plan. de surface $4\pi R^2$. L'énergie électrostatique est $W_{el}=\frac{1}{2}CV^2$. Enfin on peut dire dans cette approximation que $\vec{E}\simeq\frac{V}{z_0}\vec{e_r}$. On trouve $C=6,7\cdot10^{-2}$F, $W_{el}=4,3\cdot10^{9}$J et $E=6$V/m.
	
	\item[$\clubsuit$] Toujours dans l'analogie avec le condensateur plan, $\vec{E}=\sigma/\varepsilon\vec{e_r}$. On trouve donc $\sigma=5,3\cdot10^{-11}$C.m$^{-2}$ et $Q=4\pi R^2\sigma=24\cdot10^{3}$C.
	
	\item[$\clubsuit$] On peut dire que $E\simeq V/z_1$, où $z_1$ est l'altitude des nuages. En ordre de grandeur, on a $Z_1=1$km, donc $V_1\simeq10^8$V.
	
\end{itemize}

\end{document}
