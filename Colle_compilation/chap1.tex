\chapter*{Introduction}

\section*{Petits et gros mammifères $\bullet\circ\circ$}

Les mammifères sont des animaux à sang chaud, dont la chaleur est produite par le fonctionnement du métabolisme, qui dégage une puissance thermique $p_{th}$ par unité de volume. Cette puissance et la circulation sanguine permettent de maintenir une température $T_1$ à peu près uniforme et constante à l'intérieur du corps. Pour se prémunir contre les variations de température extérieur, les mammifères peuvent avoir une couche d'isolant (fourrure ou graisse) autour de leur corps. On souhaite connaitre la variation de cette épaisseur en fonction de la taille des mammifères. 

Pour simplifier, on considère que les mammifères sont des animaux sphériques de rayon $R$ recouverts d'une couche d'isolant d'épaisseur $e$ de capacité calorifique $c$, de masse volumique $\rho$ et de conductivité thermique $\lambda$. On notera $T(r)$ et $\vec{j}(r)$ le champ de température et le vecteur densité volumique de puissance en coordonnées sphériques, $r$ étant le rayon par rapport au centre de l'animal.