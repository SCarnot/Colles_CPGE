 \documentclass{report}
 
\usepackage[utf8]{inputenc} 
\usepackage[T1]{fontenc}      
\usepackage[top=3.5cm, bottom=3cm, left=3.0cm, right=4.0cm]{geometry}
\usepackage{graphicx}
\usepackage{amsmath}
\usepackage[activate={true,nocompatibility},final,tracking=true,kerning=true,spacing=true,factor=1100,stretch=10,shrink=10]{microtype}
\usepackage{calc}
\graphicspath{{figures/}{../figures}}

\begin{document}

\section*{Exercice 1}

\begin{itemize}
	\item[•] Avec les hypothèses : $\eta\Delta \vec{v} = \vec{grad}P$. On néglige la variation due à la gravité à l'échelle de la taille de la sphère.  
	\item[•] On cherche à connaitre le champ de pression pour connaitre la résultante des forces. 
	Liquide incompressible donc $div \vec{v}=0$ et donc $\vec{rot}(\vec{rot}(\vec{v}))=\vec{grad} (div \vec{v})-\Delta \vec{v}=-\Delta \vec{v}$, donc le gradient de pression vaut l'opposé l'expression de la vitesse donné plus haut.
	
	En choisissant une projection sur $\vec{e_r}$ ou $\vec{e_\theta}$ (ce qui revient au même) :
	\begin{align*}
		\frac{\partial P}{\partial r} = 3\eta\frac{Rv_\infty}{r^3}\cos\theta 
	\end{align*}
	
	\begin{align*}
		P = -3\eta\frac{Rv_\infty}{2r^2}\cos\theta + P_\infty
	\end{align*}
En notant $P_\infty$ la pression qui est dans le liquide loin de la sphère.

La résultante des forces de pression, selon l'axe $\vec{e_z}$, est $F_{p,z}=-P(M)d\vec{S}\vec{e_z}$. Alors :
\begin{align*}
	F_{p,z} = -\iint P_\infty R^2d\varphi \sin\theta d\theta \cos\theta + \iint 3\eta\frac{Rv_\infty}{2R^2}\cos^2\theta \sin\theta R^2d\theta d\varphi
\end{align*}
\noindent\fbox{\parbox{\linewidth-2\fboxrule-2\fboxsep}{
On trouve alors : $F_{p,z}=2\pi\eta v_\infty R$
}}

\item[•] La force de cisaillement correspond à la force de frottement due à la viscosité du fluide. Pour un élément de surface $\vec{dS}$ de la sphère, celle-ci s'écrit :
\begin{align*}
	d\vec{F_c}=\eta dS \frac{\partial v}{\partial r}\vec{e_\theta}
\end{align*}
La projection suivant $\vec{e_z}$ nous donne alors :

\noindent\fbox{\parbox{\linewidth-2\fboxrule-2\fboxsep}{
\begin{align*}
	F_{c,z}=\iint \eta \frac{\partial v}{\partial r}R^2\sin\theta d\theta d\varphi \sin\theta=\iint \eta\frac{3}{2}Rv_\infty \sin^3\theta d\theta d\varphi=4\pi\eta Rv_\infty
\end{align*}
}}
L'intégrale sur $\theta$ vaut 4/3.

\item[•] La force de trainée est la somme des deux forces précédentes. On a donc : $F_z = 6\pi\eta Rv_\infty$.
\end{itemize}

\section*{Exercice 2}

\begin{itemize}
	\item[1 - ] Comme l'écoulement est lent, et invariant selon $x$ et $y$ : l'écoulement ne dépend que de $z$. De plus, comme il est incompressible, $div(\vec{v})=0$, donc $v_z=cste=0$, car la vitesse doit être nécessairement nulle en $z=0$. L'écoulement est donc laminaire et est dirigé selon $\vec{e_z}$ (c'est une hypothèse, on suppose qu'il est dans le sens de la descente) : $\vec{v}(x,y,z)=v(z)\vec{e_z}$.
	
	\textit{NB : }avec un tel profil de vitesse, $(\vec{v}\cdot\vec{grad})\vec{v}$ est identiquement nul, mais on peut directement le négliger ce terme avec l'hypothèse de l'énoncé (écoulement lent et très visqueux).
	
	L'équation de Navier-Stokes, avec les hypothèses de l'énoncé, devient : 
\begin{align*}
	0=\eta \Delta \vec{v}+\rho\vec{g}-\vec{grad}(P)
\end{align*}
	En projetant, on obtient : 
\begin{equation}
\left\lbrace
\begin{array}{ccc}
\vec{e_x}  & : & \eta\frac{\partial^2 v}{\partial z^2}+\rho g \sin\alpha - \frac{\partial P}{\partial x}=0\\
\vec{e_z}  & : &-\rho g \cos\alpha - \frac{\partial P}{\partial z}=0\\
\end{array}\right.
\end{equation}	

En intégrant selon $\vec{e_z}$ et avec la condition au limite $P(x,z=h)=P_0$, on obtient :
\begin{align*}
	P(x,z)=P_0+\rho g\cos\alpha (h-z)
\end{align*}
En intégrant selon $\vec{e_x}$, avec la condition aux limites $\frac{\partial v}{\partial z}_{z=h}=0$ (il n'y a pas de forces de cisaillement à l'interface air/fluide) devient alors : 
\begin{align*}
	\frac{\partial v}{\partial z}=\frac{\rho g}{\eta}\sin\alpha(h-z)
\end{align*}
En intégrant une nouvelle fois, avec la condition aux limites $v(z=0)=0$ (continuité  de la vitesse avec le support) :

\noindent\fbox{\parbox{\linewidth-2\fboxrule-2\fboxsep}{
\begin{align*}
	v(z)=\frac{\rho g}{\eta}\sin\alpha z\left( h-\frac{z}{2}\right) 
\end{align*}
}}

Le profil de vitesse est parabolique.

\item[2 - ] Le débit s'écrit, en prenant comme section un carré de largeur $L\gg h$ pour que les hypothèses de l'énoncé soient valables : 
\begin{align*}
	D=\int_{y=0}^{L}dy\int_{z=0}^h dz\cdot v(z)
\end{align*}
En intégrant, on obtient :

\noindent\fbox{\parbox{\linewidth-2\fboxrule-2\fboxsep}{
\begin{align*}
	D=\frac{\rho g}{3\eta}\sin\alpha L h^3
\end{align*}
}}

\item[3 - ] La glace a une densité de 900kg.m$^3$. La vitesse proposée est la vitesse maximale de la glace, car c'est celle en surface. On a donc $\eta\simeq7.1\cdot10^{12}$Pa.s.

On obtient donc, sur une année : $V=D\Delta t\simeq4,73\cdot10^6$m$^3$, soit 4250 tonnes chaque année. 

Il faut garder à l'esprit que ce sont des ordre de grandeurs, car nous n'avons pas pris en compte les conditions aux limites sur les bords (en $y$) du canal d'écoulement du glacier, et que la vitesse est elle aussi un ordre de grandeur. 

\item[4 - ] Les conditions aux limites deviennent alors $v(z=0)=v(z=h)=0$, mais on a plus la condition sur la dérivée première de la vitesse. 
La vitesse s'écrit : $v(z)=-\frac{\rho g}{2\eta}\sin\alpha z^2 +a +b$
Avec les conditions aux limites, on trouve $b=0$ et $a=\frac{\rho g}{2\eta}\sin\alpha h$.

\noindent\fbox{\parbox{\linewidth-2\fboxrule-2\fboxsep}{
\begin{align*}
	v(z)=\frac{\rho g}{2\eta}\sin\alpha z\left( h-z\right) 
\end{align*}
}}

C'est un écoulement de Poiseuille. 

\end{itemize}

\newpage

\section*{Exercice 3}

\begin{itemize}
	\item[1 - ] Dans l'énoncé, le problème est dit invariant en $\theta$ donc la vitesse ne dépend pas de $\theta$. D'autre part l'équation de conservation s'écrit, comme le fluide est incompressible : $div(\vec{v})=0$, cad $\partial u_z/\partial z=0$, cad $u_z$ ne dépend pas de $z$. Finalement, $\vec{u}$ ne dépend que de $r$.
	
	On effectue un bilan de force sur un petit volume en coordonnées cylindrique au point $M(r,\theta,z)$. On projette directement selon $z$ :
	\begin{equation}
		rd\theta drP(z) - rd\theta drP(z+dz) - \eta\tau(r+dr)(r+dr)d\theta dz+ \eta\tau(r)rd\theta dz =0
	\end{equation}
Attention, la définition de $\tau$ implique qu'il est opposé à la variation spatiale de la vitesse. Il y a donc un signe - par rapport aux force classique de cisaillement. 
On obtient la relation voulue :

\noindent\fbox{\parbox{\linewidth-2\fboxrule-2\fboxsep}{
	\begin{align*}
		\frac{\partial P}{\partial z} + \frac{1}{r}\frac{\partial (r\tau)}{\partial r}=0
	\end{align*}}}

	\item[2 -] On commence à intégrer selon $z$. Étant donné que $u$ ne dépend pas de $z$, $\dot{\gamma}$ non plus, et $\tau$ non plus. Dès lors :
	\begin{align*}
		\int_0^Ldz P(z) = P(L)-P(0)=-\Delta P =-\frac{L}{r}\frac{\partial r\tau}{\partial r}
	\end{align*}
	On obtient donc : $\Delta P =\frac{L}{r}\frac{\partial r\tau}{\partial r}$. On intègre désormais sur $r$, et on trouve la relation demandée. Attention, durant le calcul, on intègre un $r^2$ qui fait aparaître un facteur 1/2. 
	
	\item[3 - ] Pour qu'il y ait écoulement, il faut qu'il existe une valeur de $r$ telle que $\tau>\tau_s$ (car sinon $\dot{\gamma}=0$). La plus grande valeur de $r$ est le rayon $R$, on doit alors nécessairement avoir $R>R_s$ pour espérer voir un écoulement. On a alors $\tau_s = \tau_s R/R_s$.

\noindent\fbox{\parbox{\linewidth-2\fboxrule-2\fboxsep}{	
	Cela correspond à une pression minimum de $\Delta P_{min}=\frac{2\tau_sL}{R}$.}}
	
	\item[4 - ] Comme $\Delta P > \Delta P_{min}$, on a forcément $R>R_s$. Donc pour $r>R_s$ :
	\begin{align*}
		-\frac{\partial u}{\partial r}=\frac{\tau_s}{\eta}\left( \frac{r}{R_s} - 1\right) 
	\end{align*}
	
\noindent\fbox{\parbox{\linewidth-2\fboxrule-2\fboxsep}{	
	On trouve donc pour $r>R_s$ :
	\begin{align*}
		u(r) = \frac{\Delta P}{4L\eta}(R+r-2R_s)(R-r)
	\end{align*}
	Pour $r<r_s$, on a $\dot{\gamma}=0$ donc :
	\begin{align*}
	 u(r)=u(R_s)=\frac{\Delta P}{4L\eta}(R-R_s)^2
	 \end{align*}	
}}
	Pour visualiser, l'effet, bouchon, il suffit de tracer la courbe de $u(r)$ en fonction de $r$. 
	
\end{itemize}

\section*{Exercice 4}

\begin{itemize}
	\item[1 - ] Il y a deux cas : 
	 
	\begin{itemize}
		\item[$z<0$ : ] Avec l'équation d'incompressibilité $div(\vec{v})=0$, on trouve $\frac{1}{r}\frac{\partial(r v_r}{\partial r}+\frac{\partial v_z}{\partial z} =0$. Comme dans cette partie, les lignes de courants sont parallèle à l'axe $Oz$, la vitesse es tnécessairement selon $e_z$ donc l'équation de conservation devient $\frac{\partial v_z}{\partial z} =0$, don c$v_z=cste=v_0$. Finalement,
		\begin{align*}
			\vec{v}=v_0\vec{e_z}
		\end{align*}
		
		\item[$z>0$ : ] 	La conservation du débit impose que pour toute section du tube, on ait $\iint d\vec{S}\vec{v}= \pi r^2v_z(r,z)=cste$, car $v$ ne dépend pas de $r$. En l'occurrence, pour $z=0$, on a $\pi r^2v_z(r,z)=\pi \lambda^2a^2v_0$. On a donc : 
\begin{align*}
	v_z(r,z)=\frac{a^2}{\left(a+\frac{z^2}{b} \right)^2 }v_0
\end{align*}	

Les lignes de courant sont définies par $\vec{v}\wedge d\vec{l}$. Comme $d\vec{l}=dr\vec{e_r}+dz\vec{e_z}$, donc on a :
\begin{align*}
	v_rdz-v_zdr=0\Longrightarrow \frac{v_r}{v_z}=\frac{\partial r}{\partial z}
\end{align*}

\noindent\fbox{\parbox{\linewidth-2\fboxrule-2\fboxsep}{	
	On trouve alors :
	\begin{align*}
		\vec{v}=\frac{a^2}{\left(a+\frac{z^2}{b} \right)^2 }v_0\vec{e_z}+\frac{2rz}{ab\left( 1+\frac{z^2}{ab}\right)^3 }v_0\vec{e_r}
	\end{align*}}}
	
	On peut vérifier que l'écoulement est bien incompressible. 
	\end{itemize}

\item[2 - ] Les lignes de courant s'écartent lors du passage dans la zone parabolique, l'élément de fluide se déforme de sorte à être plus fin pour respecter la conservation de la masse. 

\item[3 - ] \begin{align*}
\vec{\Omega}=\vec{rot}(\vec{v})=\left(\frac{\partial v_r}{\partial z}-\frac{\partial v_z}{\partial r} \right)\vec{e_\theta}=\frac{1-5z^2/ab}{\left( 1+z^2/ab\right)^4}\frac{2rv_0}{ab}\vec{e_\theta}\neq 0
\end{align*}

\end{itemize}

\subsubsection*{Écoulement entre deux plaques}

On peut raisonner en superposant les deux champs de vitesse comme en électromagnétisme. 

Il faut séparer le $\ln$ en deux parties pour distinguer les deux vecteurs directeurs correspondant à l'écoulement. On introduit alors $r_0$ une longueur intermédiaire.
\begin{align*}
	\vec{grad}(\phi)=av_0\left(\vec{grad}\ln\frac{r_1}{r_0}-\vec{grad}\ln\frac{r_2}{r_0} \right) 
\end{align*}
On trouve :
\begin{align*}
\vec{v}=av_0\left(\frac{\vec{e_{r_A}}}{r_A}-\frac{\vec{e_{r_B}}}{r_B} \right) 
\end{align*}

Le champs de vitesse créé par la source en A, à une distance $r_A$ de cette source peut s'écrire sous la forme :
\begin{align*}
	\vec{v_A} = v_A(r_A,z)\vec{e_{r_A}}
\end{align*} 
En effet, par invariance, $\vec{v_A}$ ne peut dépendre de $\theta$. D'autre part, comme l'écoulement est lent, il ne pourra être dirigé que selon $\vec{e_{r_A}}$.

Avec la conservation du débit, on a :
\begin{align*}
	D_e = 2\pi r_A\int_{-e/2}^{e/2}dz v(r_A,z)
\end{align*}
En notant $v_{moy}=\frac{1}{e}\int_{-e/2}^{e/2}dz v(r,z)$, qui correspond à la vitesse moyenne entre les deux plaques, on obtient :
\begin{align*}
	\vec{v_{moy}}(r_A) = \frac{D_e}{2\pi r e}\vec{e_{r_A}}
\end{align*}
La vitesse décroit en $1/r_A$. S'il n'ya pas de viscosité, on a bien $v_{moy}=v$.

Même chose pour la source en B :
\begin{align*}
	\vec{v_{moy}}(r_B) = -\frac{D_e}{2\pi r e}\vec{e_{r_B}}
\end{align*}

Au final : 
\begin{align*}
	\vec{v_{moy}} = \frac{D_e}{2\pi e}\left( \frac{1}{r_A}\vec{e_{r_A}} - \frac{1}{r_B}\vec{e_{r_B}}\right) 
\end{align*}

Les champs de vitesses sont à divergence nulle donc incompressibles.

\newpage

\section*{Exercice 5}

\begin{itemize}

\item[1 - ] Il y a invariance selon $\theta$ et selon $z$, mais aussi selon $t$ (écoulement permanent). $\vec{v}$ ne dépend donc que de $r$. 
\item[2 - ] Le bilan de matière donne : 
\begin{align*}
	\frac{1}{r}\frac{\partial(rv_r)}{\partial r} + \frac{1}{r}\frac{\partial v_\theta}{\partial \theta}+\frac{\partial v_z}{\partial z}=0
\end{align*}
Les deux derniers termes de l'équation sont nuls à cause des invariances. En intégrant le premier terme, on trouve $v_r=\frac{cste}{r}$. Comme $v_r(R_1)=0$ par conservation du débit, $\forall r, v_r=0$.

La vitesse ne dépend donc que de $r$ et a une composante uniquement selon $\vec{e_\theta}$ (d'après l'énoncé, $v_z=0$).

\item[3 - ] Le plus simple est de faire un bilan des moments selon $\vec{e_z}$ à un élément de fluide (en coordonnées cylindriques) qui est un anneau (ou tore) compris entre $r$ et $r+dr$, d'épaisseur $dz$. Les forces visqueuses s'appliquent orthoradialement alors sur les surfaces internes et externe. 

Le moment total exercé doit être nul de sorte que : 
\begin{align*}
	2\pi r dz \sigma_\theta(r)\times r - 2\pi (r+dr) dz \sigma_\theta(r+dr)\times (r+dr) = 0
\end{align*}
cad : $r^2 \sigma_\theta(r) - (r+dr)^2 \sigma_\theta(r+dr) = 0$. On en déduit que $\frac{\partial}{\partial r}r^2\sigma_\theta=0$.

D'autre part, $\sigma_\theta=\eta\left(\frac{\partial v_\theta}{\partial r}-\frac{v_\theta}{r}\right) =\eta r \frac{\partial}{\partial r}\frac{v_\theta}{r} $. On trouve le résultant voulu :

\noindent\fbox{\parbox{\linewidth-2\fboxrule-2\fboxsep}{	
	\begin{align*}
		\frac{\partial}{\partial r}\left(r^3 \frac{\partial}{\partial r}\frac{v_\theta}{r} \right) =0
	\end{align*}}}
	
	\item[4 - ] On intègre la relaiton précédente et on trouve :
	\begin{align*}
		v_\theta = \frac{A}{r} + Br
	\end{align*}
	Les conditions aux limites sont : $v_\theta(r=R_{1,2})=R_{1,2}\Omega_{1,2}$. On obtient alors : 
	
	\noindent\fbox{\parbox{\linewidth-2\fboxrule-2\fboxsep}{	
	\begin{align*}
		v_\theta  = \frac{\Omega_2R_2^2-\Omega_1R_1^2}{R_2^2-R_1^2}r+\frac{(\Omega_1-\Omega_2)R_2^2R_1^2}{R_2^2-R_1^2}\frac{1}{r}
	\end{align*}}}

\item[5 - ] Le couple exercé par le cylindre de rayon $R_1$ s'exprime comme la résultantes des contraintes de cisaillement dues à la viscosité :
\begin{align*}
	M_{\eta}(R_1)=\iint R_1d\theta dz R_1\sigma_\theta(R_1)= 2\pi R_1^2L\sigma_\theta(R_1)
\end{align*}
Or, $\sigma_\theta(R_1) = -2B\eta/R_1^2$ donc on trouve que :

\noindent\fbox{\parbox{\linewidth-2\fboxrule-2\fboxsep}{	
\begin{align*}
	M_{\eta}(R_1) = 4\pi\eta\frac{(\Omega_1-\Omega_2)R_2^2R_1^2}{R_2^2-R_1^2}
\end{align*}}}

L'angle du ressort est donc : $\theta = M_{\eta}(R_1)/k$. On peut donc mesurer la viscosité du fluide.

\end{itemize}

\newpage

\section{Exercice 6}

\begin{itemize}

	\item[1 - ] Comme $2\vec{\Omega}=\vec{rot}(\vec{v})$, le théorème de Stockes nous donne sur un cercle $C$ fermé de centre $O$ de rayon $r$ : 
	\begin{align*}
		\oint_C \vec{v}(r,\theta,z)\cdot rd\theta\vec{e_\theta}=2\iint_{S_C} d\vec{S}\cdot\vec{\Omega}
	\end{align*}
	
\noindent\fbox{\parbox{\linewidth-2\fboxrule-2\fboxsep}{	
Pour $r<a$, on a :
	\begin{align*}
		\oint_C \vec{v}(r,\theta,z)\cdot rd\theta\vec{e_\theta}=2\pi r^2 \Omega
	\end{align*}
Pour $r>a$, on a :
	\begin{align*}
		\oint_C \vec{v}(r,\theta,z)\cdot rd\theta\vec{e_\theta}=2\pi a^2 \Omega
	\end{align*}}}

Il s'agit du même problème que celui du fil infini de rayon parcouru par une densité volumique de courant uniforme.

\item[2 - ] Les intégrales précédentes permettent de trouver que :

\noindent\fbox{\parbox{\linewidth-2\fboxrule-2\fboxsep}{	
Pour $r<a$, on a :
	\begin{align*}
		\vec{v}(r)\vec{e_\theta}= r \Omega\vec{e_\theta}
	\end{align*}
Pour $r>a$, on a :
	\begin{align*}
		\vec{v}(r)\vec{e_\theta}= \frac{a^2}{r} \Omega\vec{e_\theta}
	\end{align*}}}
	
\item[3 - ] 	
	
Pour $r<a$, l'équation de Navier-Stockes donne : 
\begin{align*}
\vec{grad}\frac{v^2}{2}+2\vec{\Omega}\wedge\vec{v}=-g\vec{e_z}-\frac{1}{\rho}\vec{grad}P
\end{align*}
	Le premier terme est l'accélération centrifuge et vaut $\Omega r^2\vec{e_r}$, le second vaut $-2\Omega r^2\vec{e_r}$. En projetant sur $r$ et $z$ et en intégrant la pression, on obtient :
	\begin{align*}
		P(r,z) = k + \frac{1}{2}\rho\Omega^2r^2-\rho g z
	\end{align*}
La constante $k$ sera à déterminer avec la continuité de la pression.


Pour $r<a$, l'équation de Navier-Stockes donne : 
\begin{align*}
\vec{grad}\frac{v^2}{2}=-g\vec{e_z}-\frac{1}{\rho}\vec{grad}P
\end{align*}
	Le premier terme (l'accélération centrifuge) vaut $-\frac{a^4\Omega^2}{r^3}\vec{e_r}$. En projetant sur $r$ et $z$ et en intégrant la pression, on obtient :
	\begin{align*}
		P(r,z) = P_0 -\rho g z+ \frac{\rho a^4\Omega^2}{2r^2}
	\end{align*}
	On sait en effet que la pression pour $r\longrightarrow\infty$ est égale à la pression atmosphérique $P_0-\rho gz$.
	
La continuité de $P(r,z)$ en $a$ donne $k=P_0-\rho \Omega^2a^2$.

Au final :
\begin{align*}
	\left\lbrace
\begin{array}{ccc}
r<a & : & P(r,z) = P_0 -\rho g z +\rho\Omega^2(\frac{r^2}{2} -a^2) \\
 & & \\
r>a & : & P(r,z) = P_0 -\rho g z+ \frac{\rho a^4\Omega^2}{2r^2} \\
\end{array}\right.
\end{align*}

\item[4 - ] La pression est minimale en $r=0$, et surtout elle ne peut pas être négative. Dans le cas extrême où $P(r=0)=0$, on a $P_0-\rho\Omega^2a^2=0$, cad $a\Omega=\sqrt{P_0/\rho}=v(r=a)=v_{max}$.
Pour une pression $P_0=$1bar, on a une vitesse maximale de $277$m.s$^{-1}$ ! Ce qui correspond à des vents de 1000km.h$^{-1}$, ce qui n'est pas observable. D'autre phénomènes interviennent avec la compressibilité de l'air, qui est négligée ici, avec des turbulences fortes.

\item[5 - ] Avec le même raisonnement précédent, on a $a\Omega=$180km.h$^{-1}$=50m.s$^{-1}$. On suppose que lorsque la tornade passe sur le bâtiment, la pression à l'intérieur est restée à $P_0$. Il s'ne suit une différence de pression de : $\Delta P = P_0 -P_{min}=-\rho a^2\Omega^2$=3250Pa, cad une force de 325kg par m$^2$. Le toit s'envole.

\end{itemize}

\newpage

\section*{Exercice 7}

\begin{itemize}

	\item[1 - ] On a en tout point du tube $v=-\dot{h}$. L'écoulement n'est pas permanent ! Donc :
	\begin{align*}
		\rho\frac{\partial v}{\partial t} +\frac{\rho}{2}\vec{grad(}v^2)+\rho\cdot \vec{rot}(\vec{v})\wedge\vec{v} =\rho\vec{g}-\vec{grad}P
	\end{align*}

Soit $A$ et $B$ les points respectivement en haut du liquide et à la sortie du tuyau. On calcule la circulation entre les points $A$ et $B$ pour chaque terme :
\begin{align*}
	\int_A^B\rho\frac{\partial \vec{v}}{\partial t}\cdot d\vec{l}=-\ddot{h}(t)\rho(L+h(t))
\end{align*}

Les termes $\frac{\rho}{2}\vec{grad(}v^2)$ et $\rho\cdot \vec{rot}(\vec{v})\wedge\vec{v}$ sont nuls. 

\begin{align*}
	\int_A^B\rho\vec{g}d\vec{l}=g\rho h(t)
\end{align*}

La pression est égale à la pression atmosphérique à l'entrée et à la sortie du tuyau :
\begin{align*}
	\int_A^B\vec{grad}(P)d\vec{l}=P_1-P_2
\end{align*}

On en déduit :
\begin{align*}
	\ddot{h}=-\frac{gh}{L+h}
\end{align*}

En multipliant par $\dot{h}$ de chaque côté, on a :
\begin{align*}
	\dot{h}\ddot{h}=-g\left(\dot{h} - \frac{L\dot{h}}{L+h}\right) 
\end{align*}

En intégrant, on obtient : 
\begin{align*}
	\frac{\dot{h}^2}{2}=g(h_0-h)-gL\ln\left(\frac{h_0+L}{h+L} \right) 
\end{align*}

On obtient alors :
\begin{align*}
	v(h)=\sqrt{2g\left( h_0-h-L\ln\left(\frac{h_0+L}{h+L}  \right)\right)  }
\end{align*}

Pour $L\longrightarrow0$, $v^2\approx2g(h_0-h)$. Cela correspond à la colonne de liquide en chute libre.

\item[2 - ] Dans la branche verticale, pour un point $M$ à une altitude $z$ :

\begin{align*}
	\int_A^M\rho\frac{\partial \vec{v}}{\partial t}\cdot d\vec{l}=-\ddot{h}(t)\rho(h(t)-z)
\end{align*}

\begin{align*}
	\int_A^M\rho\vec{g}\cdot d\vec{l}=g\rho(h(t)-z)
\end{align*}

\begin{align*}
	\int_A^M\vec{grad}(P)\cdot d\vec{l}=P_0-P(M,t)
\end{align*}

On a alors :
\begin{align*}
	P(M,t)=P_0+\rho\left[ h(t)-z \right] \left[g+\ddot{h} \right] 
\end{align*}
Soit :

\noindent\fbox{\parbox{\linewidth-2\fboxrule-2\fboxsep}{	
\begin{align*}
	P(M,t)=P_0+\rho g \frac{L\left[ h(t)-z \right]}{L+h(t)}
\end{align*}}}

Dans la branche horizontale, pour un point $M$ à la position $x$ :
\begin{align*}
	\int_M^B\rho\frac{\partial \vec{v}}{\partial t}\cdot d\vec{l}=-\ddot{h}(t)\rho(L-x)
\end{align*}

\begin{align*}
	\int_M^B\rho\vec{g}\cdot d\vec{l}=0
\end{align*}

\begin{align*}
	\int_M^B\vec{grad}(P)\cdot d\vec{l}=P(M,t)-P_0
\end{align*}

On a alors :
\begin{align*}
	P(M,t)=P_0+\rho\ddot{h}(L-x)
\end{align*}
Soit :

\noindent\fbox{\parbox{\linewidth-2\fboxrule-2\fboxsep}{	
\begin{align*}
	P(M,t)=P_0+\rho g \frac{h(t)\left[ L-x \right]}{L+h(t)}
\end{align*}}}

\end{itemize}

\section*{Exercice 8}

\begin{itemize}

	\item[$\clubsuit$] La relation fondamentale s'écrit :
	\begin{align*}
		\rho\frac{D\vec{v}}{Dt}=\rho\vec{g}-\vec{grad}P-2\rho\vec{\Omega}\wedge\vec{v}
	\end{align*}
Il n'y a pas de viscosité, car fluide parfait.

	\item[$\clubsuit$] Si l'air est immobile, alors $\vec{v}=\vec{0}$, donc $\rho\vec{g}=\vec{grad}P$. Comme le gaz est parfait, $\rho=PM/RT$ et donc :
	\begin{align*}
		P_{eq}(z)=P_0\exp(-z/H)
	\end{align*}
	où $H=RT/Mg\approx8,5$km. L'atmosphère fait environ $5H$, soit 42km. Comme cette épaisseur est très petite par rapport au rayon de la terre, on considèrera que $v_z=0$.
	
	\item[$\clubsuit$] On a : $\rho\vec{g}-\vec{grad}P=\rho\vec{g}-\vec{grad}P_{eq}-\vec{grad}p$. L'équation de la dynamique devient alors :
		\begin{align*}
		\rho\frac{D\vec{v}}{Dt}=-\vec{grad}p-2\rho\vec{\Omega}\wedge\vec{v}
	\end{align*}
	
	\item[$\clubsuit$] Le terme convectif peut s'estimer comme $U^2/L$ et le terme de coriolis $2U\Omega$. Le nombre de Rossby s'écrit donc :
	\begin{align*}
		Ro = \frac{U}{2L\Omega}
	\end{align*}
	On trouve $Ro=7\cdot10^{-2}$. Le terme de Coriolis est donc prépondérant. Alors : 
			\begin{align*}
		0=-\vec{grad}p-2\rho\vec{\Omega}\wedge\vec{v}
	\end{align*}
	
	\item[$\clubsuit$] On a $\vec{\Omega}=(-\cos\lambda\vec{e_x}+\sin\lambda\vec{e_z})\Omega$. Alors : 
	\begin{equation}
		-2\rho\vec{\Omega}\wedge\vec{v}=-2\rho\Omega(-\sin\lambda v_y\vec{e_x}+\sin\lambda v_x\vec{e_y}-\cos\lambda v_y\vec{e_z})
	\end{equation}
	
	Pour faire apparaître l'expression de la vitesse seule, on applique le produit vectoriel $\vec{e_z}\wedge$ :
	\begin{equation}
		-\vec{e_z}\wedge2\rho\vec{\Omega}\wedge\vec{v}=-2\rho\Omega(\sin\lambda v_x\vec{e_x}+\sin\lambda v_y\vec{e_y})=-2\rho\Omega \sin\lambda\vec{v}
	\end{equation}
	
	Au final, on obtient : 
		\begin{equation}
		\vec{v}=\frac{\vec{e_z}\wedge\vec{grad}(p)}{2\rho\Omega\sin\lambda}
	\end{equation}
	
	\item[$\clubsuit$] On retrouve facilement le sens des vents avec le gradient moyen de pression. /!$\backslash$ Le gradient de pression s'écrit $\vec{grad}(p)=-\frac{1}{R_T}\frac{dP}{d\lambda}\vec{e_x}$, il augmente de 0$^\circ$ à 20$^\circ$, diminue de 20$^\circ$ et 60$^\circ$ puis remonte de 60 à 80$^\circ$. 
	
	En conséquence, les vents soufflent d'est en ouest entre 0$^\circ$ à 20$^\circ$, d'ouest en est entre 20$^\circ$ et 60$^\circ$ et d'est en ouest de 60$^\circ$ à 80$^\circ$. Pas de vents au-delà.
	
	\item[$\clubsuit$] A la latitude 45$^\circ$, $\vec{grad}(p)=-\frac{1}{R_T}\frac{dP}{d\lambda}\vec{e_x}\simeq 1,01\cdot10^{-3}$Pa, $\rho=PM/RT\simeq1,16$kg.m$^{-3}$. On a alors $v\simeq6$m.s$^{-1}$.
	
\end{itemize}

\end{document}
