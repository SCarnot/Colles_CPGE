 \documentclass{report}
 
\usepackage[utf8]{inputenc} 
\usepackage[T1]{fontenc}      
\usepackage[top=3.5cm, bottom=3cm, left=3.0cm, right=4.0cm]{geometry}
\usepackage{graphicx}
\usepackage{amsmath}
\usepackage[activate={true,nocompatibility},final,tracking=true,kerning=true,spacing=true,factor=1100,stretch=10,shrink=10]{microtype}
\usepackage{calc}
\graphicspath{{figures/}{../figures}}

\begin{document}

\section*{Exercice 1}

\begin{itemize}
	\item[•] Avec les hypothèses : $\eta\Delta \vec{v} = \vec{grad}P$. On néglige la variation due à la gravité à l'échelle de la taille de la sphère.  
	\item[•] On cherche à connaitre le champ de pression pour connaitre la résultante des forces. 
	Liquide incompressible donc $div \vec{v}=0$ et donc $\vec{rot}(\vec{rot}(\vec{v}))=\vec{grad} (div \vec{v})-\Delta \vec{v}=-\Delta \vec{v}$, donc le gradient de pression vaut l'opposé l'expression de la vitesse donné plus haut.
	
	En choisissant une projection sur $\vec{e_r}$ ou $\vec{e_\theta}$ (ce qui revient au même) :
	\begin{align*}
		\frac{\partial P}{\partial r} = 3\eta\frac{Rv_\infty}{r^3}\cos\theta 
	\end{align*}
	
	\begin{align*}
		P = -3\eta\frac{Rv_\infty}{2r^2}\cos\theta + P_\infty
	\end{align*}
En notant $P_\infty$ la pression qui est dans le liquide loin de la sphère.

La résultante des forces de pression, selon l'axe $\vec{e_z}$, est $F_{p,z}=-P(M)d\vec{S}\vec{e_z}$. Alors :
\begin{align*}
	F_{p,z} = -\iint P_\infty R^2d\varphi \sin\theta d\theta \cos\theta + \iint 3\eta\frac{Rv_\infty}{2R^2}\cos^2\theta \sin\theta R^2d\theta d\varphi
\end{align*}
\noindent\fbox{\parbox{\linewidth-2\fboxrule-2\fboxsep}{
On trouve alors : $F_{p,z}=2\pi\eta v_\infty R$
}}

\item[•] La force de cisaillement correspond à la force de frottement due à la viscosité du fluide. Pour un élément de surface $\vec{dS}$ de la sphère, celle-ci s'écrit :
\begin{align*}
	d\vec{F_c}=\eta dS \frac{\partial v}{\partial r}\vec{e_\theta}
\end{align*}
La projection suivant $\vec{e_z}$ nous donne alors :

\noindent\fbox{\parbox{\linewidth-2\fboxrule-2\fboxsep}{
\begin{align*}
	F_{c,z}=\iint \eta \frac{\partial v}{\partial r}R^2\sin\theta d\theta d\varphi \sin\theta=\iint \eta\frac{3}{2}Rv_\infty \sin^3\theta d\theta d\varphi=4\pi\eta Rv_\infty
\end{align*}
}}
L'intégrale sur $\theta$ vaut 4/3.

\item[•] La force de trainée est la somme des deux forces précédentes. On a donc : $F_z = 6\pi\eta Rv_\infty$.
\end{itemize}

\section*{Exercice 2}

\begin{itemize}
	\item[1 - ] Comme l'écoulement est lent, et invariant selon $x$ et $y$ : l'écoulement ne dépend que de $z$. De plus, comme il est incompressible, $div(\vec{v})=0$, donc $v_z=cste=0$, car la vitesse doit être nécessairement nulle en $z=0$. L'écoulement est donc laminaire et est dirigé selon $\vec{e_z}$ (c'est une hypothèse, on suppose qu'il est dans le sens de la descente) : $\vec{v}(x,y,z)=v(z)\vec{e_z}$.
	
	\textit{NB : }avec un tel profil de vitesse, $(\vec{v}\cdot\vec{grad})\vec{v}$ est identiquement nul, mais on peut directement le négliger ce terme avec l'hypothèse de l'énoncé (écoulement lent et très visqueux).
	
	L'équation de Navier-Stokes, avec les hypothèses de l'énoncé, devient : 
\begin{align*}
	0=\eta \Delta \vec{v}+\rho\vec{g}-\vec{grad}(P)
\end{align*}
	En projetant, on obtient : 
\begin{equation}
\left\lbrace
\begin{array}{ccc}
\vec{e_x}  & : & \eta\frac{\partial^2 v}{\partial z^2}+\rho g \sin\alpha - \frac{\partial P}{\partial x}=0\\
\vec{e_z}  & : &-\rho g \cos\alpha - \frac{\partial P}{\partial z}=0\\
\end{array}\right.
\end{equation}	

En intégrant selon $\vec{e_z}$ et avec la condition au limite $P(x,z=h)=P_0$, on obtient :
\begin{align*}
	P(x,z)=P_0+\rho g\cos\alpha (h-z)
\end{align*}
En intégrant selon $\vec{e_x}$, avec la condition aux limites $\frac{\partial v}{\partial z}_{z=h}=0$ (il n'y a pas de forces de cisaillement à l'interface air/fluide) devient alors : 
\begin{align*}
	\frac{\partial v}{\partial z}=\frac{\rho g}{\eta}\sin\alpha(h-z)
\end{align*}
En intégrant une nouvelle fois, avec la condition aux limites $v(z=0)=0$ (continuité  de la vitesse avec le support) :

\noindent\fbox{\parbox{\linewidth-2\fboxrule-2\fboxsep}{
\begin{align*}
	v(z)=\frac{\rho g}{\eta}\sin\alpha z\left( h-\frac{z}{2}\right) 
\end{align*}
}}

Le profil de vitesse est parabolique.

\item[2 - ] Le débit s'écrit, en prenant comme section un carré de largeur $L\gg h$ pour que les hypothèses de l'énoncé soient valables : 
\begin{align*}
	D=\int_{y=0}^{L}dy\int_{z=0}^h dz\cdot v(z)
\end{align*}
En intégrant, on obtient :

\noindent\fbox{\parbox{\linewidth-2\fboxrule-2\fboxsep}{
\begin{align*}
	D=\frac{\rho g}{3\eta}\sin\alpha L h^3
\end{align*}
}}

\item[3 - ] La glace a une densité de 900kg.m$^3$. La vitesse proposée est la vitesse maximale de la glace, car c'est celle en surface. On a donc $\eta\simeq7.1\cdot10^{12}$Pa.s.

On obtient donc, sur une année : $V=D\Delta t\simeq4,73\cdot10^6$m$^3$, soit 4250 tonnes chaque année. 

Il faut garder à l'esprit que ce sont des ordre de grandeurs, car nous n'avons pas pris en compte les conditions aux limites sur les bords (en $y$) du canal d'écoulement du glacier, et que la vitesse est elle aussi un ordre de grandeur. 

\item[4 - ] Les conditions aux limites deviennent alors $v(z=0)=v(z=h)=0$, mais on a plus la condition sur la dérivée première de la vitesse. 
La vitesse s'écrit : $v(z)=-\frac{\rho g}{2\eta}\sin\alpha z^2 +a +b$
Avec les conditions aux limites, on trouve $b=0$ et $a=\frac{\rho g}{2\eta}\sin\alpha h$.

\noindent\fbox{\parbox{\linewidth-2\fboxrule-2\fboxsep}{
\begin{align*}
	v(z)=\frac{\rho g}{2\eta}\sin\alpha z\left( h-z\right) 
\end{align*}
}}

C'est un écoulement de Poiseuille. 

\end{itemize}

\end{document}
