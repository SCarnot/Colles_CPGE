 \documentclass{report}
 
\usepackage[utf8]{inputenc} 
\usepackage[T1]{fontenc}      
\usepackage[top=2.0cm, bottom=3cm, left=3.0cm, right=3.0cm]{geometry}
\usepackage{graphicx}
\usepackage{wrapfig}
\usepackage{amsmath,esint }
\usepackage{amssymb}
\graphicspath{{figures/}{../figures}}

\newcommand*\dif{\mathop{}\!\mathrm{d}}
\newcommand*\diver{\mathop{}\!\mathrm{div}}
\newcommand*\grad{\mathop{}\!\mathrm{grad}}

\begin{document}

\section*{Courants de Foucault}

\begin{itemize}

	\item[$\diamondsuit$] Grand classique de l'induction de Neumann. Le champ magnétique extérieur induit un champ électromoteur $\vec{E}=E(r)\vec{e}_\theta$ : en effet, par l'équation de Maxwell-Faraday, le champ éléctromoteur tourne autour du champ magnétique qui l'induit. D'autre part, les invariances et symétries imposent une dépendance seulement en $r$. Toujours avec cette même équation, on a :
	\begin{align*}
		\oint_\Gamma \vec{E}\cdot d\vec{l} = -\frac{\partial \phi_B}{\partial t}
	\end{align*}	
En prenant pour contour $\Gamma$ un cercle de rayon $r$, de centre $O$ dont la normale est dirigée par $\vec{e}_z$ :
	\begin{align*}
		2\pi r\times E(r)= B_0\omega\sin(\omega t)\pi r^2
	\end{align*}
Les courants induits sont donc :
	\begin{align*}
		 \vec{j}(r)= \frac{\gamma B_0\omega r}{2}\sin(\omega t) \vec{e}_\theta
	\end{align*}

	
	\item[$\diamondsuit$] En se plaçant toujours dans l'ARQS, on peut calculer le champ magnétique $\vec{B}_{ind}$ induit par ces courants. Par invariance et symétrie, ce champ magnétique ne dépend que de $r$ et est dirigé suivant $\vec{e}_z$.
	
	 On utilise le théorème d'Ampère, en prenant pour contour $\Gamma$ le rectangle $ABCD$ suivant : $AB$ est un segment de longueur $h$ confondu avec l'axe $Oz$, $BC$ un segment colinéaire à l'axe $\vec{e}_r$ de longueur $r$ et $CD$, $DA$ de sorte à refermer le rectangle. Alors : 
	\begin{align*}
		h(B(r=0)-B(r))=\frac{\mu_0\gamma B_0\omega }{2}\sin(\omega t)\times h\int_0^r r'dr'
	\end{align*}

On trouve ainsi : 
\begin{align*}
	\vec{B}(r) = \left[ B(r=0) - \frac{\mu_0\gamma B_0\omega r^2}{4}\sin(\omega t)\right] \vec{e}_z
\end{align*}

\textbf{\textit{A partir de là, je ne suis plus du tout sûr de ce que je dis !}}

Le champ magnétique $B(r=0)$ correspond au champ magnétique initial, cad $B_0$. 
\begin{align*}
	\vec{B}(r) = B_0\left[cos(\omega t)  - \frac{\mu_0\gamma \omega r^2}{4}\sin(\omega t)\right] \vec{e}_z
\end{align*}
On a donc bien un champ induit qui s'oppose au champ magnétique générateur $B_0$ : la valeur du champ diminue jusqu'à : $B_0\left[cos(\omega t)  - \frac{\mu_0\gamma \omega a^2}{4}\sin(\omega t)\right]$., ce qui est la valeur du champ dans tout l'espace pour $r>a$.

Problème : cette valeur du champ n'est pas bornée, le champ induit peut être plus fort que le champ générateur !!! Surtout si $\gamma$ est suffisament fort :'(

\end{itemize}

\section*{Induction mutuelle entre deux circuits}

\begin{itemize}

	\item[$\clubsuit$] L'inductance propre $L$ correspond au flux du champ magnétique créé par un circuit fermé sur lui même. L'induction mutuelle $M$ correspond au flux du champ magnétique d'une bobine à travers l'autre bobine. Comme le champ magnétique créé par une bobine décroit nécessairement avec la distance, l'inductance mutuelle (au carré) est nécessairement plus petite que le produit des inductances propres : $M<L_1L_2$. Elle peut être au mieux égale dans le cas où toute les lignes de champs d'un circuit sont canalisées dans l'autre. On peut montrer cette inégalité à partir de l'énergie magnétique d'un tel ensemble : 
	\begin{align*}
		\varepsilon_M=\frac{1}{2}L_1i_1^2+\frac{1}{2}L_2i_2^2+Mi_1i_2
	\end{align*}
	Comme $\varepsilon_M>0$, en étudiant la positivité du polynôme en $x=i_2/i_1$, on trouve la condition $M<L_1L_2$.
	
	\item[$\clubsuit$] On trouve rapidement que :
	
	\begin{align*}
	\left\lbrace
	\begin{array}{ccc}
	i_1 + L_1C_1\frac{\dif^2 i_1}{\mathrm{dt}^2}+C_1M\frac{\dif^2 i_2}{\mathrm{dt}^2}=0\\
	\\
	i_2 + L_2C_2\frac{\dif^2 i_2}{\mathrm{dt}^2}+C_2M\frac{\dif^2 i_1}{\mathrm{dt}^2}=0\\
	\end{array}\right.
	\end{align*}		
	
	\item[$\clubsuit$] Dans le cas $L_1=L_2=L$, on pose $\alpha=\frac{M}{L}$ et $\omega_0^2=1/LC$.
	
	\begin{align*}
	\left\lbrace
	\begin{array}{ccc}
	\omega_0^2i_1+\frac{\dif^2 i_1}{\mathrm{dt}^2}+\alpha\frac{\dif^2 i_2}{\mathrm{dt}^2}=0\\
	\\
	\omega_0^2i_2+\frac{\dif^2 i_2}{\mathrm{dt}^2}+\alpha\frac{\dif^2 i_1}{\mathrm{dt}^2}=0\\
	\end{array}\right.
	\end{align*}		
	
	En posant $I=i_1+i_2$ et $i=i_1-i_2$, on tombe sur les équations différentielles :
	\begin{align*}
	\left\lbrace
	\begin{array}{ccc}
	\omega_0^2I+(1+\alpha)\frac{\dif^2 I}{\mathrm{dt}^2}=0\\
	\\
	\omega_0^2i+(1-\alpha)\frac{\dif^2 i}{\mathrm{dt}^2}=0\\
	\end{array}\right.
	\end{align*}		
	
	On trouve alors : $I(t)=A\cos(\omega_1t+\phi)$ et $i(t)=B\cos(\omega_2t+\varphi)$, $\omega_1=\omega_0/\sqrt{1+\alpha}$, $\omega_2=\omega_0/\sqrt{1-\alpha}$ avec $A$, $B$, $\phi$ et $\varphi$ des constantes d'intégration. $I$ et $i$ sont appelés "modes propres" du systèmes.
	
	On peut ainsi retrouver les expressions de $i_1$ et $i_2$ comme des superpositions de ces deux modes propres : $i_1=(i+I)/2$ et $i_2=(I-i)/2$. On a alors :
		\begin{align*}
	\left\lbrace
	\begin{array}{ccc}
	i_1(t)=\frac{A}{2}\cos(\omega_1t+\phi)+\frac{B}{2}\cos(\omega_2t+\varphi)\\
	\\
	i_2(t)=\frac{A}{2}\cos(\omega_1t+\phi)-\frac{B}{2}\cos(\omega_2t+\varphi)\\
	\end{array}\right.
	\end{align*}		
	
	\item[$\clubsuit$] Le spectre contient alors deux fréquences $\omega_1$ et $\omega_2$, centrées autour de $\omega_0$. Dans le cas où le couplage devient très faible, $\alpha\longrightarrow0$ et alors $\omega_1\simeq\omega_0(1-\frac{1}{2}\alpha$ et $\omega_2\simeq\omega_0(1+\frac{1}{2}\alpha$. On voit que le spectre se scinde effectivement en deux pulsations, partant de $\omega_0$ et "écartés" en fréquence de $\delta\omega=\omega_0\alpha$. De manière générale, tout système harmonique couplé à un autre système harmonique voit sa fréquence propre se dédoubler autour de la sienne, avec une fréquence propre proche de la sienne, et une autre égale à la différence des deux. 
	
	On voit apparaître le phénomène de battements, visible dans la solution de $i_1$ dans le cas de faible couplage ($\alpha\longrightarrow0$), avec $A=B=i_0$ :
	\begin{align*}
		i_1(t)=i_0\cos(\omega_0t+\phi)\cos(\delta\omega t)
	\end{align*}
	Le terme $\cos(\delta\omega t)$ fait office de "battement" (pulsation lente) cad d'une enveloppe sur la variation rapide $\cos(\omega_0t+\phi)$.
	
	\item[$\clubsuit$] Si les condensateurs sont déchargés à $t=0$, la tension à leur borne est nulle, et à celle des bobines aussi. On a donc : $\frac{\mathrm{d}i_1}{\mathrm{dt}}=\frac{\mathrm{d}i_2}{\mathrm{dt}}=0$ et donc $\phi=\varphi=0$.
	
	Pour obtenir qu'une seule fréquence, il faut soit $A=0$, soit $B=0$. Cela correspond respectivement à $i_1(t=0)=-i_2(t=0)$ et à $i_1(t=0)=i_2(t=0)$. On parle de mode symétrique et de mode antisymétrique. 
	
	\item[$\clubsuit$] La loi des mailles donne
		\begin{align*}
	\left\lbrace
	\begin{array}{ccc}
	\frac{q_1}{C}=L\frac{\mathrm{d}i_1}{\mathrm{dt}}+M\frac{\mathrm{d}i_2}{\mathrm{dt}}\\
	\\
	\frac{q_2}{C}=L\frac{\mathrm{d}i_2}{\mathrm{dt}}+M\frac{\mathrm{d}i_1}{\mathrm{dt}}\\
	\end{array}\right.
	\end{align*}		
En multipliant par $i_1$ la première équation, et en utilisant que $i_1=-\frac{\mathrm{d}q_1}{\mathrm{dt}}$ (et même chose pour le circuit 2), on trouve que :
\begin{align*}
	\frac{\mathrm{d}}{\mathrm{dt}}\left(\frac{q_1^2}{2C}+\frac{q_2^2}{2C}+\frac{Li_1^2}{2}+\frac{Li_2^2}{2}+Mi_1i_2 \right) =0
\end{align*}
Cad l'énergie du système 1+2 se conserve.
	 
\end{itemize}

\section*{Induction mutuelle entre deux lignes bifilaires}

\begin{itemize}

	\item[$\clubsuit$] La géométrie étant identique pour les deux circuits, l'inductance est strictement la même. On peut la calculer pour le circuit 1, en notant $A$ et $B$ le centre de chaque cable. A ce moment là, le champ magnétique créé par ces fils est :
	\begin{align*}
		\vec{B}_{A/B}(M) = \frac{\mu_0i_1}{2\pi r_{A/B}}\vec{e}_{\theta, A/B}
	\end{align*}

	\item[$\clubsuit$] Déterminer les équations différentielles satisfaites par $i_1(t)$ et $i_2(t)$.
	
	\item[$\clubsuit$] En proposant un changement de fonction bien choisi avec $i_1(t)$ et $i_2(t)$, trouver la solution générale pour $i_1$ et $i_2$. Pourquoi parle t-on de modes propres ?
	
	\item[$\clubsuit$] Quelle est l'allure du spectre de $i_1$ ? Dans le cas d'un faible couplage $M$, montrer que le spectre se scinde en deux harmoniques centrées autour de $\omega_0$, séparées en fréquence de $\delta\omega$, que l'on déterminera.  
	
	\item[$\clubsuit$] On suppose qu'à $t=0$, les deux condensateurs sont déchargés. Pour quelles valeurs de $i_1(t=0)$ et $i_2(t=0)$ n'y a t-il qu'une fréquence dans le spectre de $i_1$ et $i_2$ ?
	
	\item[$\clubsuit$] Réaliser un bilan de puissance électrique et commenter. 	
	
\end{itemize}	

\subsubsection*{Questions bonus}

\begin{itemize}
		
	\item[$\bigstar$] Calculer l'inductance mutuelle $M$ entre les deux circuits. A t-on bien $M<L$ ? 
	
	\item[$\bigstar$] Calculer la capacité $C$ de chaque ligne. 
	
\end{itemize}

\newpage

\section*{Courants de Foucault dans un cylindre en rotation}

\subsubsection*{Champ axial}

\begin{itemize}

	\item[$\diamondsuit$] La force de Lorentz qui s'exerce sur un électron de conduction est $\vec{f}=-e(\vec{E}+\vec{v}\wedge\vec{B})$. Sa vitesse étant radiale, il subit alors une force dirigée selon $\vec{e_r}$. Les électrons se déplacent sur les bords du cylindre mais ne peuvent pas "boucler" pour former une boucle de courant : il n'y a donc pas de courant de Foucault en régime permanent.
	
	\item[$\diamondsuit$] Les électrons vont se déplacer sur les bords du cylindre, jusqu'à que leur répartition créée un champ électrique opposé à la force de Lorentz, puis qui le compense, de sorte à ce que $\vec{f}=\vec{0}=-e(\vec{E}+\vec{v}\wedge\vec{B})$. On peut voir l'effet du champ magnétique comme un champ électrique "moteur" $\vec{E_m}=\vec{v}\wedge\vec{B}=r\omega B\vec{e_\theta}$. L'équilibre des forces donne un champ électrique créé par la distribution de charge comme $\vec{E}=\vec{E_m}=-r\omega B\vec{e_\theta}$.
	
	Comme $\diver(\vec{E})=\frac{\rho}{\varepsilon_0}$, on a :
	\begin{align*}
		\frac{1}{r}\frac{\partial E_r}{\partial r} 	= \frac{\rho}{\varepsilon_0}
	\end{align*}
	Et alors :
	\begin{align*}
		\rho = -2\omega B\varepsilon_0
	\end{align*}
	Les électrons sont partis sur la surface, la charge volumique créé en volume est créé par les ions du cristal fixes. Par neutralité de la charge, la charge dans le volume est compensée par une charge surfacique : $\rho h\pi R^2=-\sigma 2\pi Rh$. On a donc :
	\begin{align*}
		\sigma = \varepsilon_0 B\omega R
	\end{align*}
 
\end{itemize}

\subsubsection*{Champ transverse}	
	
	On applique désormais un champ magnétique uniforme $\vec{B}=B_{0}\vec{e_{x}}$ transverse à l'axe de rotation.
	
\begin{itemize}
	
		\item[$\square$] Même chose que dans la première question : avec la force de Lorentz, on trouve que les électrons de conduction sont soumis à une force dirigée selon $\vec{e_z}$. Il y a des courants qui se forment en boucle sur la longueur du cylindre. Ces courants vont dissiper de la puissance par effet Joule et exercer un couple résistant selon l'axe $\vec{e_z}$ (c'est aussi la loi de Lenz : Les courants de Foucault induits vont s'opposer au champ magnétique extérieur qui leur a donné naissance, et la force de Laplace associée exerce un couple résistif).

		\item[$\square$] On a $\vec{j}(r,\theta)=-\vec{j}(r,\theta+\pi)$. En effet, les causes de l'apparition des courants de Foucault (le champ magnétique et la rotation) sont les mêmes en $(r,\theta)$ et en $(r,\theta+\pi)$, les effets sont donc les mêmes : on a donc $\parallel\vec{j}(r,\theta)\parallel=\parallel\vec{j}(r,\theta+\pi)\parallel$. Comme la vitesse est opposée entre ces deux points, la norme doit l'être aussi. On peut aussi le justifier avec la fermeture des boucle de courant.
		
		\item[$\square$] On prend comme contour $\Gamma$ une boucle rectangulaire avec pour médiane l'axe $\vec{e_z}$, de largeur $2r$, de longueur la hauteur du cylindre. Comme on a la relation $\vec{j}=\gamma\vec{E}$, le champ électrique est colinéaire au champ de courant. Alors, le théorème de Maxwell-Faraday, avec le théorème de Stockes donne :
		\begin{align*}			
			\oint_\Gamma \vec{E}\cdot\dif\vec{l}=-\frac{\mathrm{d}}{\mathrm{dt}}\iint_{S_{\Gamma}}\vec{B}\cdot\dif\vec{S}	
		\end{align*}
		Comme précisé dans l'énoncé, on néglige ce qu'il se passe là où le courant se "retourne", près des extrémités supérieure et inférieure du cylindre. Dans l'intégrale, seules comptent les contributions verticales :
			\begin{align*}			
			-\oint_\Gamma E(r,\theta)\dif z=-\frac{\mathrm{d}}{\mathrm{dt}}\iint_{S_{\Gamma}}B\dif r\dif z\vec{e_x}\cdot\vec{e_\theta}	
		\end{align*}
		\textit{NB} : le signe $-$ qui apparait est dû est l'orientation du contour par rapport à la géométrie du problème (règle de la main droite).
		On trouve, en sachant que $\frac{\mathrm{d}}{\mathrm{dt}}\vec{e_x}\cdot\vec{e_\theta}=-\omega\cos\theta$ :
		\begin{align*}			
			2hE(r,\theta)=-2rhB\omega\cos\theta
		\end{align*}
		On trouve donc : 
		\begin{align*}			
			\vec{E}(r,\theta)=-rB\omega\cos\theta\vec{e_z}
		\end{align*}
		\item[$\square$] On a la relation $\vec{j}=\gamma\vec{E}$ :
		\begin{align*}			
			\vec{j}(r,\theta)=-\gamma rB\omega\cos\theta\vec{e_z}
		\end{align*}
		
		\item[$\square$] La puissance dissipée est $dP=\vec{j}\cdot\vec{E}d\tau$.
		On trouve :
		\begin{align*}
		P=\gamma B^2\omega^2\int_{z=0}^h \dif z\int_{\theta=0}^{2\pi}\int_{r=0}^{a}r^3\cos^2\theta\dif r\dif\theta=\frac{\pi}{4}\gamma B^2ha^4\omega
		\end{align*}
		
		\item[$\square$] On peut utiliser la relation $P=\omega\Gamma$.
		
\end{itemize}

\section*{Canon électromagnétique}

\subsubsection*{Cas statique}

\begin{itemize}

	\item[$\heartsuit$] $\Phi_B=L(x_0)I(t)$, et avec la loi de Faraday :
	\begin{align*}
		e=-L(x_0)\frac{\mathrm{d}I(t)}{\mathrm{dt}}
	\end{align*}
	
	\item[$\heartsuit$] En faisant une loi des mailles puis en multipliant par l'intensité :
	\begin{align*}
		P_G=R(x_0)I^2(t)-eI(t)=R(x_0)I^2(t)+\frac{1}{2}L(x_0)\frac{\mathrm{d}} {\mathrm{dt}}I^2(t)
	\end{align*}
	Le second terme $\frac{1}{2}L(x_0)\frac{\mathrm{d}} {\mathrm{dt}}I^2(t)$ correspond à l'énergie magnétique.

\end{itemize}

\subsubsection*{Cas mobile}

\begin{itemize}

	\item[$\triangle$] Première justification : avec la force de laplace. Deuxième justification : avec la loi de Lenz, le barreau aura tendance à agrandir le circuit pour s'opposer à la variation du flux de $\vec{B}$ créé par le circuit lui-même.
	
	La puissance fournie par le générateur s'écrit :
	
	\begin{align*}
		P_G&=R(x)I^2(t)-e(t)I(t) \\
		&= R(x)I^2(t)+I(t)\frac{\mathrm{d}I(t)L(x)}{\mathrm{d}t} \\
		&= R(x)I^2(t)+I^2(t)\frac{\mathrm{d}L(x)}{\mathrm{d}t}+L(x)I(t)\frac{\mathrm{d}I(t)}{\mathrm{d}t} \\
		&= R(x)I^2(t)+I^2(t)\dot{x}\frac{\mathrm{d}L(x)}{\mathrm{dx}}+L(x)I(t)\frac{\mathrm{d}I(t)}{\mathrm{d}t}
	\end{align*}
	
Le premier terme est le terme d'effet Joule. L'autre terme correspond à la puissance mécanique et magnétique. 

\item[$\triangle$] On a donc :
\begin{align*}	
	I^2(t)\dot{x}\frac{\mathrm{d}L(x)}{\mathrm{dx}}+L(x)I(t)\frac{\mathrm{d}I(t)}{\mathrm{d}t} &= \frac{\mathrm{d}\varepsilon_{m}}{\mathrm{d}t} + P_{meca} \\
	&= \frac{1}{2}I^2(t)\dot{x}\frac{\mathrm{d}L(x)}{\mathrm{dx}} + L(x)I(t)\frac{\mathrm{d}I(t)}{\mathrm{d}t}+ P_{meca}
\end{align*}	
	
	\begin{align*}
		 P_{meca} = \frac{1}{2}I^2(t)\dot{x}\frac{\mathrm{d}L(x)}{\mathrm{dx}} 
	\end{align*}
	
	La puissance mécanique fait bien intervenir la vitesse du barreau, soit :
	\begin{align*}
		P_{meca}=\frac{1}{2}\dot{x}(t)\frac{\mathrm{d}L(x)}{\mathrm{dx}}I^2(t)
	\end{align*}	
La force est alors :
	\begin{align*}
		F=\frac{1}{2}\frac{\mathrm{d}L(x)}{\mathrm{dx}}I^2(t)
	\end{align*}	

\end{itemize}

\subsubsection*{Étude du mouvement}

On suppose que le générateur est constitué d'une dynamo couplée à une bobine d'inductance $L_0$ et de résistance $R_0$. Tant que l'interrupteur $C$ est fermé, la dynamo impose un fort courant $I_0$ dans la bobine. A $t=0$, où l'on ouvre $C$, le courant s'écoule alors dans les rails et accélère le barreau. 

On suppose par ailleurs que $L(x)=L'x$ et $R(x)=R'x$, où $L'$ et $R'$ sont respectivement l'inductance et la résistance linéique.

\begin{itemize}

	\item[$\diamondsuit$] L'inductance totale du circuit est $L_{tot}=L_0+L'x$ donc $e=-\frac{\mathrm{d}L_{tot}I(t)}{\mathrm{dt}}=-I(t)L'\dot{x}-(L_0+L'x)\frac{\mathrm{d}I(t)}{\mathrm{dt}}$. On a alors :
	\begin{align*}
		-I(t)L'\dot{x}-(L_0+L'x)\frac{\mathrm{d}I(t)}{\mathrm{dt}} = (R_0+xR')I(t)
	\end{align*}
	
	\item[$\diamondsuit$] En utilisant la formule de la force trouvée précédemment :
	\begin{align*}
		M\ddot{x}(t)=\frac{1}{2}I^2L'
	\end{align*}
	
	\item[$\diamondsuit$] à $t=0$, on a $x=x_0$ et $\dot{x}=0$ par inertie, de même $I(0)=I_0$. Les solutions stationnaires impliquent $\ddot{x}=0$ cad $I=0$, mais c'est incompatible avec les conditions initiales. 
	
	\item[$\diamondsuit$] On a un circuit $(r,L)$ en série, de temps caractéristique $\tau\simeq L_0/R$, si $L_0$ est très grand, ce temps sera très grand devant le temps d'éjection du barreau et $I(t)$ n'aura quasiment pas varié. 
	
	A ce moment là :
	\begin{align*}		
		x(t)=\frac{1}{2}\frac{L'I_0^2}{M} t^2	
	\end{align*}

\end{itemize}

\section*{Chauffage par induction}

\begin{itemize}

	\item[$\clubsuit$] La plaque est dans un champ magnétique variable, il y a donc une force électromotrice induite dans le conducteur générant des courants de Foucault. Au vu des symétrie et de l'orientation du champ $\vec{B}$, les courants seront de la forme $\vec{j}=j(r,t)\vec{e_\theta}$.
	
	\item[$\clubsuit$] Le théorème de Faraday s'écrit :
	\begin{align*}
		\oint_\Gamma\vec{E}\dif\vec{l}=-\frac{d}{dt}\oiint_{S_\Gamma}\vec{B}\dif\vec{S}
	\end{align*}
	En prenant comme contour $\Gamma$ un cercle de rayon $r$ centré en $O$, et en utilisant la loi $\vec{j}=\gamma\vec{E}$, on obtient :
	\begin{align*}
	\begin{array}{ccc}
	r<a & : & \frac{2\pi rj(r,t)}{\sigma}=-\frac{\dif B(t)}{\dif t}\pi r^2\\
	a<r<b & : & \frac{2\pi rj(r,t)}{\sigma}=-\frac{\dif B(t)}{\dif t}\pi a^2\\
	\end{array}
	\end{align*}		
On a alors :
	\begin{align*}
	\begin{array}{ccc}
	r<a & : & j(r,t)=\frac{\sigma\omega B_m r\sin(\omega t)}{2}\\
	a<r<b & : &  j(r,t)=\frac{\sigma\omega B_m a^2\sin(\omega t)}{2r}\\
	\end{array}
	\end{align*}		
		
	\item[$\clubsuit$] La puissance volumique est donnée par $p_{Joule}= j^2(r,t)/\sigma$. On a alors :
	\begin{align*}
		P_{Joule}=2\pi e\int_0^a \frac{\sigma\omega^2 B_m^2 r^2\sin^2(\omega t)}{4}rdr+2\pi e\int_a^b \frac{\sigma\omega^2 B_m^2 a^4\sin^2(\omega t)}{4r^2}rdr
	\end{align*}
	On trouve :
	\begin{align*}
		P_{Joule}=\frac{\pi e a^4\sigma\omega^2 B_m^2 r^2\sin^2(\omega t)}{2}\left[\frac{1}{4} + \ln\left(\frac{b}{a} \right)  \right] 
	\end{align*}
	
	\item[$\clubsuit$] Le champ magnétique est créé à partir de bobines enroulées en dessous de la plaque, parcourues par des courants sinusoïdaux. La dissipation de puissance par effet Joule se fait directement dans le casserole et non dans la plaque.
	
	\item[$\clubsuit$] $\langle P_{Joule}\rangle\simeq5$kW
	
	\item[$\diamond$] Le champ induit correspond à ce que l'auto induction de la plaque est faible en regard de l'induction mutuelle entre les deux circuits. En effet, le flux du champ induit est donné par $\oiint B_{ind.}\dif S=L_Pi_{Foucault}$ et du champ extérieur par $\oiint B_{ext.}\dif S=Mi_{Foucault}$. Si $B_{ind.}\ll B_{ext.}$, comme les géométries sont presque les mêmes, alors $L_P\ll M$.
	
\end{itemize}

\end{document}
