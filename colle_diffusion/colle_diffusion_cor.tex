 \documentclass{report}
 
\usepackage[utf8]{inputenc} 
\usepackage[T1]{fontenc}      
\usepackage[top=2.0cm, bottom=3cm, left=3.0cm, right=3.0cm]{geometry}
\usepackage{graphicx}
\usepackage{wrapfig}
\usepackage{amsmath,esint }
\usepackage{amssymb}
\graphicspath{{figures/}{../figures}}

\newcommand*\dif{\mathop{}\!\mathrm{d}}
\newcommand*\diver{\mathop{}\!\mathrm{div}}
\newcommand*\grad{\mathop{}\!\mathrm{grad}}

\begin{document}

\section*{Lac gelé}

\begin{itemize}

	\item[$\ast$] En redémontrant l'équation de la chaleur, on trouverait :
	\begin{align*}
		\frac{\partial^2 T}{\partial z^2} = \frac{\rho c_g}{\lambda}\frac{\partial T}{\partial t} \simeq0
	\end{align*}
	car $c_g\simeq0$ dans le cadre de l'ARQS. On a alors :
	\begin{align*}
		T(z) = \frac{T_F-T_0(t)}{\xi(t)}z+T_0(t)
	\end{align*}
	
	\item[$\ast$] Par conservation de l'énergie, la densité de flux thermique dans la glace $j_g(z,t)=-\lambda\frac{\partial T}{\partial z}$ est égale à la densité de flux thermique à l'interface $j_a$ donné par la loi de Newton :
	\begin{align*}
		& j_g(z=0,t)=j_a \\
		\Rightarrow & -\frac{\lambda}{\xi(t)}(T_F-T_0(t))=-h(T_A-T_0(t))
	\end{align*}
	On en déduit :
	\begin{align*}
		T_0(t)=\frac{T_F+\frac{h\xi(t)}{\lambda}T_A}{1+\frac{h\xi(t)}{\lambda}}
	\end{align*}
	
	\item[$\ast$] On considère un volume d'eau $dV=S\times dz$ se transformant en glace durant un temps $dt$. Comme le front de galce avance à la vitesse $\dot{\xi}$, on a $dz=\dot{\xi}dt$. Ce volume dégage une énergie $dQ=l_f\rho_g\dot{\xi}dtS$ lors de sa transformation, qui est évacuée à travers le flux thermique dans la glace $dQ = -Sdt\times j_g(z=\xi,t)$. Le signe $-$ correspond au fait que l'énergie est évacuée vers l'atmosphère, donc selon $-\vec{e}_z$.
	
	On a alors : 
	\begin{align*}
		&\frac{\lambda}{\xi(t)}(T_F-T_0(t))=l_F\rho_g\dot{\xi}(t) \\
		\Rightarrow & T_F-\frac{T_F+\frac{h\xi(t)}{\lambda}T_A}{1+\frac{h\xi(t)}{\lambda}} =\frac{l_F\rho_g}{\lambda} \xi(t)\dot{\xi}(t) \\
		\Rightarrow & \frac{h}{l_F\rho_g}(T_F-T_A)=\left(1+\frac{h}{\lambda}\xi(t) \right) \dot{\xi}(t)
	\end{align*}
	
	On a donc :
	\begin{align*}
		\left( 1+\frac{\xi(t)}{l_0}\right) \dot{\xi(t)}=v_0
	\end{align*}
	Avec $l_0=\frac{\lambda}{h}$ et $v_0=\frac{h}{l_F\rho_g}(T_F-T_A)$. On peut vérifier l'homogénéité, qui est bien respectée. On remarque par ailleurs que $v_0=\dot{\xi}(t=0)$, qui correspond à la vitesse d'avancée du front de glace au début de la glaciation.
	
	\item[$\ast$] $l_0= 0,05m$ et de $v_0=1,27\cdot10^{-6}$m/s. On définit un temps caractéristique $\tau_0$ comme tout simplement $\tau_0=l_0/V_0=39\cdot10^{3}$s.
	
	\item[$\ast$] On intègre la relation précédente :
	\begin{align*}
			&v_0=\frac{l_0}{2}\frac{d}{dt}\left(1+\frac{\xi(t)}{l_0} \right)^2 \\
			\Rightarrow & \frac{2t}{\tau_0}=\left(1+\frac{\xi(t)}{l_0} \right)^2+C
	\end{align*}
	En tenant compte des conditions initiales, $\xi(0)=0$, on trouve $C=-1$.
	Donc, en résolvant l'équation du second degré et en prenant la racine positive, on obtient :
	\begin{align*}
		\xi(t) = l_0\left( \sqrt{1+\frac{2t}{\tau_0}}-1\right) 
	\end{align*}
	Et de même : 
	\begin{align*}
		T_0(t)=T_A+\frac{T_F-T_A}{\sqrt{1+\frac{2t}{\tau_0}}}
	\end{align*}
	
	Pour qu'il y ait 5 cm de glace, soit une épaisseur $\xi(t)=l_0$, il faut un temps $t=\frac{3}{2}\tau_0\simeq59\cdot10^{3}$s, soit environ 16h30. Cela montre bien que la vitesse de glaciation diminue, si elle se maintenait à $v_0$, il faudrait un temps $\tau_0$ pour geler ces 5cm, soit environ 11h. 

\end{itemize}

\section*{Transfert thermique et entropie}

\begin{itemize}

	\item[$\triangleright$] Il faut d'abord obtenir l'équation de la chaleur lorsqu'une source de chaleur est présente dans le volume. L'équation de conservation (qui n'en est plus une !) devient :
	\begin{align*}
		c\rho\frac{\partial T}{\partial t} = -\frac{\partial j}{\partial x} + p_{vol}
	\end{align*}
	où $p_{vol} = \rho I^2/S^2$ est la puissance volumique dissipée par effet Joule. 
	L'équation de la chaleur devient donc :
	\begin{align*}
		\frac{\partial T}{\partial t} = \frac{\lambda}{c\rho}\frac{\partial^2 T}{\partial x^2} + \frac{p_{vol}}{c\rho}
	\end{align*}	
	En régime permanent :
	\begin{align*}
	  \lambda\frac{\partial^2 T}{\partial x^2} = - p_{vol}
	\end{align*}	
	En intégrant, on obtient :
	\begin{align*}
		T(x) = -\frac{p_{vol}x^2}{2\lambda}+Ax+B
	\end{align*}
	Avec les CL $T(0)=T_1$ et $T(L)=T_2$, on trouve :
	\begin{align*}
		T(x) = \frac{p_{vol}}{2\lambda}x(L-x)+\frac{T_2-T_1}{L}x+T_1
	\end{align*}
	\item[$\triangleright$] Il y a plusieurs façon de répondre à la question. Je préfère utiliser la condition où $T'(L)=0$, cad que la puissance électrique chauffe suffisament pour tordre la courbe de température de sorte à ce que la dérivée deviennent nulle en $x=L$ (en considérant que $T_2>T_1$).
	Comme on a :
	\begin{align*}
		T'(x) = \frac{p_{vol}}{2\lambda}(L-2x)+\frac{T_2-T_1}{L}
	\end{align*}
	La condition $T'(L)=0$ donne alors : 
	\begin{align*}
		p_{vol} = 2\lambda\frac{T_2-T_1}{L^2}
	\end{align*}
	
	\item[$\triangleright$] On effectue un bilan d'entropie sur le système constitué d'une tranche de barre entre $x$ et $x+dx$ du barreau durant un temps $dt$ :
	\begin{align*}
		dS = \delta s_e(x) - \delta s_e(x+dx) +\delta s_c
	\end{align*}
Il s'agit, pour l'entropie échangée, du même raisonnement que dans le cas d'une machine thermique avec deux sources extérieures, un échange d'netropie en $x$ et un autre en $x+dx$. $\delta s_c$ est l'entropie crée durant $dt$

En RP, $dS = 0$ et $\delta s_e(x)=\frac{\delta Q(x)}{T(x)}=Sdt\frac{j(x)}{T(x)}$. Alors :
\begin{align*}
	& Sdt\frac{j(x)}{T(x)} - Sdt\frac{j(x+dx)}{T(x+dx)} +\delta s_c = 0 \\
	& \delta s_c =  -S\lambda dt\left( \frac{1}{T(x+dx)}\frac{\partial T}{\partial x}(x+dx)\right) + S\lambda dt\left( \frac{1}{T(x)}\frac{\partial T}{\partial x}(x)\right) \\
	& \frac{ \delta s_c}{Sdxdt}=-\lambda\frac{\partial}{\partial x}\left(\frac{1}{T(x)}\frac{\partial T}{\partial x} \right) = -\frac{\lambda}{T(x)}\frac{\partial^2 T}{\partial x^2} + \frac{\lambda}{T^2(x)}\left( \frac{\partial T}{\partial x} \right) ^2
\end{align*}
L'entropie crée par unité de volum eet de temps est donc :
\begin{align*}
	\frac{s_{c,vol}}{dt}  = \frac{\rho I^2}{S^2 T}+ \frac{\lambda}{T^2(x)}\left( \frac{\partial T}{\partial x} \right) ^2 >0
\end{align*}
Le premier terme correspond à l'entropie crée par effet Joule, le second par les transferts thermiques.
		
	\item[$\triangleright$] On peut réutiliser les résultats de la question précédente, pour $I=0$ :
		\begin{align*}
		T(x) = \frac{T_2-T_1}{L}x+T_1
	\end{align*}
	
	\item[$\triangleright$] Il y a plusieurs façon de calculer la température finale. Pour ma part, j'utilise la conservation de l'énergie thermique $U$ de la barre lors de la transformation. Comme celle-ci est totalement isolée, on a :
	$U(t=0)=U_\infty=SL\mu c T_\infty$ où $\mu$ est la masse volumique de la barre.
	Or, l'énergie interne à $t=0$ est la somme de toutes les énergies internes du barreau d'épaisseurs $dx$ :
	\begin{align*}
		U(t=0) &= c\mu S\int_0^LdxT(x) \\
		&=c\mu SL\frac{T_1+T_2}{2}
	\end{align*}
	donc $T_\infty = \frac{T_1+T_2}{2}$, on trouve bien que la température finale correspond à la moyenne des empératures extrêmes. 
	Pour le calcul de l'entropie, on se base sur la variation d'entropie d'un solide lors d'une transformation d'une température à une autre. Plus précisément, un élément $Sdx$ de la barre à l'abscisse $x$ passe de la témprature $T(x)$ à $T_\infty$. Sa variation d'entropie est donc :
	\begin{align*}
		\Delta (\delta S) = \mu c S dx \ln\left(\frac{T_\infty}{T(x)} \right) 
	\end{align*}
	Donc pour l'ensemble de la barre :
	\begin{align*}
		\Delta S &= -\mu c S \int_0^Ldx \ln\left(\frac{T(x)}{T_\infty} \right) \\
		& = -\mu c S\frac{L}{T_2-T_1}\int_{T_1}^{T_2}dT\ln\left(\frac{T}{T_\infty}\right)
	\end{align*}
Finalement : 
\begin{align*}
\Delta S = Mc \left( 1 + \frac{T_1}{T_2-T_1}\ln\left(\frac{2T_1}{T_1+T_2}\right) - \frac{T_2}{T_2-T_1}\ln\left(\frac{2T_2}{T_1+T_2}\right)\right)
\end{align*}

\section*{Température dans une planète naine}

\begin{itemize}

	\item[$\circledcirc$] La concentration de thorium est de $c=10\times10^{-6}\times\mu$=27g.m$^{-3}$, soit une quantité $n=cN_A/M=7,00\times10^{22}$ atomes de thorium par m$^3$. La puissance peut être estimée par l'énergie $\varepsilon$ d'une désintégration divisée par le temps de demi-vie $\tau$ (ce qui correspond peu ou prou à l'activité nucléaire, un facteur $\ln2$ près), soit $P_r=\frac{\varepsilon n}{\tau}=8,90\times10^{-8}$ W.m$^{-3}$.
	
	\item[$\circledcirc$] On effectue un bilan d'enthalpie entre $r$ et $r+dr$ :
	\begin{align*}
		& r^2dr\sin\theta d\theta d\phi\times \mu c_p\times[T(r,t+dt)-T(r,t)]= \\
		& r^2\sin\theta d\theta d\phi\times j(r,t)dt-(r+dr)^2\sin\theta d\theta d\phi\times j(r+dr,t)dt + P_rdt r^2dr\sin\theta d\theta d\phi
	\end{align*}
	On a donc :
	\begin{align*}
		 r^2 \mu c_p\times\frac{\partial T}{\partial t}(r,t)= \frac{\partial}{\partial r} \left( r^2j(r,t)\right) + r^2\times P_r
	\end{align*}	
	Et alors, avec la loi de Fourier $j=-\lambda\frac{\partial T}{\partial r}$ :
	\begin{align*}
		\frac{\partial T}{\partial t}=\frac{D}{r^2}\frac{\partial }{\partial r}\left( r^2\frac{\partial T}{\partial r}\right)  + \kappa		
	\end{align*}
Avec $\kappa=P_r/(\mu c_p)$ et de $D=\lambda/(\mu c_p)$.

	\item[$\circledcirc$] Le temps caractéristique de diffusion thermique est estimé comme $\tau_d \simeq L^2/D=L^2\mu c_p/\lambda=6,09\times10^{15}$ s, soit 0,19 milliard d'années. C'est long mais toujours bien inférieur à 14 milliards d'années, qui est le temps caractéristique de décroissance radioactive du thorium. La planète a donc le temps d'être à tout instant thermalisée avec l'extérieur. 
	
	\item[$\circledcirc$] On peut donc estimer que le terme $\partial T/\partial t$ est nul. L'équation de diffusion devient :
	\begin{align*}
		\frac{D}{r^2}\frac{\partial }{\partial r}\left( r^2\frac{\partial T}{\partial r}\right) =  -\kappa		
	\end{align*}
	L'intégration fait apparaître deux constantes d'intégration, $A$ et $B$ :
	\begin{align*}
		T(r) = -\frac{\kappa}{6D}r^2-\frac{A}{r}+B
	\end{align*}
	La température étant définie en tout point de la planête, y compris en $r=0$, on a nécessairement $A=0$. 

	 La planète perd de l'énergie thermique par rayonnement, qui part dans l'espace depuis sa surface. La puissance thermique associée à ce rayonnement suit la loi de Stefan-Boltzmann $\phi=\sigma T_s^4$, où $\phi$ est la puissance rayonnée par unité de surface à la surface de la planète, $\sigma=5,67\times10^{-8}$ W.m$^{-2}$.K$^{-4}$ une constante et $T_s$ la température à la surface de l'astre.

	 \item[$\circledcirc$] La loi de Stefan-Boltzmann donne une seconde CL : $j(r=R)=-\lambda\frac{\partial T}{\partial r}(r=R)=\sigma T^4(r=R)$. On a donc :
	 \begin{align*}
	 	\lambda\frac{\kappa}{3D}R=\sigma\left(B-\frac{\kappa}{6D}R^2\right)^4
	 \end{align*}
	 et donc :
	 \begin{align*}
	 	B = \sqrt[4]{\frac{P_r R}{3\sigma}}+\frac{\kappa}{6D}R^2
	 \end{align*}
	 Finalement :
	 \begin{align*}
	 	T(r) = \frac{\kappa}{6D}\left( R^2 - r^2\right) +\sqrt[4]{\frac{P_r R}{3\sigma}}
	 \end{align*}
	 
	 \item[$\circledcirc$] Pour $T(0)=\sqrt[4]{\frac{P_r R}{3\sigma}}+\frac{\kappa}{6D}R^2\simeq63$ K et $T(R)=\sqrt[4]{\frac{P_r R}{3\sigma}}\simeq 20$ K.

\end{itemize}	

\end{itemize}

\end{document}